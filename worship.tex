\chapter{Directory for the Worship of God}

\section{Preamble}  This directory\index{Worship:directory:directory ``f20!} seeks to avoid the extremes of being so rigorous as to be viewed as a liturgy\index{Worship:public:liturgy:not imposed ``f20!}, or so loose\index{Worship:directory:balance between liturgy and looseness ``f20!} that it serves no purpose. There are some matters it contains that may be inconsistent\index{Worship:directory:possible inconsistency with scruples of some ``f20!} with the scruples\index{Office bearers:subscription:scruples ``f20!} of every office bearer\index{Office bearers:office-bearers ``f20!} due to the various opinions concerning the relationship between the gifts\index{Office bearers:functions vs gifts ``f20!}, functions, and\index{Office bearers:functions ``f20!} offices\index{Office bearers:offices ``f20!} that Christ has instituted for the Church. Nevertheless, we hope that it will be received in the spirit of the directory produced by the Westminster Assembly\index{Westminster:Assembly ``f20!}, which wrote:  \textit{{``Albeit we have not expressed in the Directory every minute particular which is or might be either laid aside or retained among us, as comely and useful in practice, yet we trust that none will be so tenacious of old customs not expressly forbidden, or so averse from good examples although new, in matters of lesser}}\index{Worship:directory:matters of lesser importance ``f20!}\textit{ consequence, as to insist upon their liberty}\index{Worship:directory:not insist on liberty ``f20!}\textit{ of retaining the one or refusing the other, because not specified in the Directory, but be studious to please others rather than themselves.'' }

\section{Introduction}  God has established and revealed in the Bible how to publicly\index{Worship:regulated by the Bible ``f20!} worship him.\footnote{\ Hebrews 12:28\index{58 Hebrews:ch.  12, v.  28, pg. ``f30!}; Genesis 4:3--8\index{01 Genesis:ch.   4, v.   3--8, pg. ``f30!} with Hebrews 11:4\index{58 Hebrews:ch.  11, v.   4, pg. ``f30!}; Exodus 20:2--6\index{02 Exodus:ch.  20, v.   2--6, pg. ``f30!}; Exodus 25:40\index{02 Exodus:ch.  25, v.  40, pg. ``f30!}; Leviticus 10:1-20\index{03 Leviticus:ch.  10, v.   1-20, pg. ``f30!}; 1 Samuel 15:22\index{09 1 Samuel:ch.  15, v.  22, pg. ``f30!}; Matthew 15:1--14\index{40 Matthew:ch.  15, v.   1--14, pg. ``f30!}; Deuteronomy 12:8\index{05 Deuteronomy:ch.  12, v.   8, pg. ``f30!}; Deuteronomy 17:3\index{05 Deuteronomy:ch.  17, v.   3, pg. ``f30!}; Colossians 2:8-10\index{51 Colossians:ch.   2, v.   8-10, pg. ``f30!}, 16-23\index{51 Colossians:ch.   2, v.   8-10, 16-23, pg. ``f30!}.} To add to or subtract from what God has commanded is not acceptable, because it introduces human traditions\index{Worship:no human traditions ``f20!} and leads to unnecessary divisions in the Church.\footnote{\ 1 Corinthians 4:6\index{46 1 Corinthians:ch.   4, v.   6, pg. ``f30!}; 1 Corinthians 11:2\index{46 1 Corinthians:ch.  11, v.   2, pg. ``f30!}, 16\index{46 1 Corinthians:ch.  11, v.   2, 16, pg. ``f30!}; Matthew 15:9\index{40 Matthew:ch.  15, v.   9, pg. ``f30!}; Colossians 2:8-10\index{51 Colossians:ch.   2, v.   8-10, pg. ``f30!}, 16-23\index{51 Colossians:ch.   2, v.   8-10, 16-23, pg. ``f30!}; Ephesians 4:2-3\index{49 Ephesians:ch.   4, v.   2-3, pg. ``f30!}.} Therefore, the Lord's Day\index{Sabbath:Lord's Day:worship ``f20!} public worship\index{Worship:public:Lord's Day ``f20!} is to be comprised of only those elements\index{Worship:elements:elements ``f20!} and ordinances\index{Worship:ordinances ``f20!} instituted by the\index{Worship:elements:instituted:by Jesus ``f20!} Lord Jesus Christ as given through the prophets and apostles.\footnote{\ Jeremiah 19:4-5\index{24 Jeremiah:ch.  19, v.   4-5, pg. ``f30!}; Colossians 2:18-23\index{51 Colossians:ch.   2, v.  18-23, pg. ``f30!}.} Under the New Covenant\index{Covenant:new ``f20!} instituted by the Lord Jesus Christ, public worship is no longer connected with the physical temple, Levitical priests, animal sacrifices, Levitical choirs\index{Worship:public:choirs ``f20!}, or the use of musical\index{musical instruments ``f20!} instruments, that were all commanded by the Lord through Moses\index{Moses ``f20!} and David in the Old Covenant\index{Covenant:old ``f20!}.\footnote{\ Leviticus 23:4, 37\index{03 Leviticus:ch.  23, v.   4, 37, pg. ``f30!}; Hebrews 9:1-10, 24\index{58 Hebrews:ch.   9, v.   1-10, 24, pg. ``f30!}; 1 Chronicles 6:31\index{13 1 Chronicles:ch.   6, v.  31, pg. ``f30!}; 1 Chronicles 15:16\index{13 1 Chronicles:ch.  15, v.  16, pg. ``f30!}; 1 Chronicles 23:5\index{13 1 Chronicles:ch.  23, v.   5, pg. ``f30!}; 1 Chronicles 25:1-6\index{13 1 Chronicles:ch.  25, v.   1-6, pg. ``f30!}; 2 Chronicles 5:12\index{14 2 Chronicles:ch.   5, v.  12, pg. ``f30!}; 2 Chronicles 7:6\index{14 2 Chronicles:ch.   7, v.   6, pg. ``f30!}; 2 Chronicles 29:25-30\index{14 2 Chronicles:ch.  29, v.  25-30, pg. ``f30!}.} These types of ordinances\index{Covenant:old, ordinances ``f20!} have passed away, leaving the\index{Worship:simplicity:without Old Covenant ordinances ``f20!} Church with the simplicity of congregational worship\index{Worship:simplicity:in spirit and truth ``f20!} ``in spirit and\index{Worship:in spirit and truth ``f20!} truth.{''}\footnote{\ John 4:21, 23-24\index{43 John:ch.   4, v.  21, 23-24, pg. ``f30!}; Hebrews 12:27-29\index{58 Hebrews:ch.  12, v.  27-29, pg. ``f30!}; Acts 7:48\index{44 Acts:ch.   7, v.  48, pg. ``f30!}; Acts 17:25\index{44 Acts:ch.  17, v.  25, pg. ``f30!}.}   When it comes to public worship\index{Worship:public:Lord's Day ``f20!}, the Church needs to consider:    

\begin{enumerate}
      \item There are the \textit{elements}\index{Worship:elements:elements ``f20!}\textit{,} that is, those things which are essential to the public worship\index{Worship:public:Lord's Day ``f20!}; and there are \textit{circumstances} of public worship, such as where to worship\index{Worship:locations ``f20!} and at what time, etc.
      \item The \textit{elements}\index{Worship:elements:elements ``f20!} are clearly expressed either by command and example in Scripture or by good and necessary consequence may be deduced from Scripture.
      \item The \textit{circumstances} are not explicitly set forth in Scripture and are decided upon at the discretion\index{Session:discretionary items ``f20!} of a church session, remembering that all things should be done decently and in order.\footnote{\ 1 Corinthians 11:13-14\index{46 1 Corinthians:ch.  11, v.  13-14, pg. ``f30!}; 1 Corinthians 14:26\index{46 1 Corinthians:ch.  14, v.  26, pg. ``f30!}, 40\index{46 1 Corinthians:ch.  14, v.  26, 40, pg. ``f30!}.}  
\end{enumerate}

\begin{outerlst}[left=0pt,labelsep=0pt]
\item
\restartlist{innerlst}
\section{Chapter 1 -- Lord's Day Public Worship Services}  \index{Worship:public:Lord's Day ``f20!}\index{Worship:public:services ``f20!}

\subsection{General Principles of Public Worship} \index{Worship:public:Lord's Day ``f20!}

\begin{innerlst}[wide, labelwidth=!, labelindent=0pt, labelsep=4pt]
    \item The Lord expressly calls\index{Worship:commanded by the Lord ``f20!} and commands his people to draw near to him, for the sacred\index{Worship:sacred duty ``f20!} duty and high\index{Worship:public:Lord's Day ``f20!} privilege of public worship each Lord's\index{Sabbath:Lord's Day:worship ``f20!} Day, as it is a Sabbath\index{Sabbath:Sabbath ``f20!} of solemn rest\index{Sabbath:solemn rest ``f20!} and holy convocation\index{Sabbath:day for holy convocation ``f20!} to him.\footnote{\ Leviticus 23:3\index{03 Leviticus:ch.  23, v.   3, pg. ``f30!}; Acts 20:7\index{44 Acts:ch.  20, v.   7, pg. ``f30!}; 1 Corinthians 11:17-18, 33-34\index{46 1 Corinthians:ch.  11, v.  17-18, 33-34, pg. ``f30!}; 1 Corinthians 14:23, 26\index{46 1 Corinthians:ch.  14, v.  23, 26, pg. ``f30!}; Hebrews 10:25\index{58 Hebrews:ch.  10, v.  25, pg. ``f30!}.}
    \item Under the Old Testament\index{Worship:public:morning and evening:Old Testament ``f20!}, God established a pattern of Sabbath\index{Sabbath:Sabbath ``f20!} day morning and evening worship\index{Worship:public:morning and evening:biblical pattern ``f20!}.\footnote{\ Numbers 28:1-10\index{04 Numbers:ch.  28, v.   1-10, pg. ``f30!}.} In the New Testament, the\index{Worship:public:morning and evening:New Testament ``f20!} Lord Jesus seemingly supports\index{Worship:public:morning and evening:assemble ``f20!} this same pattern by meeting with his disciples both in the morning and the evening on the Lord's Day\index{Sabbath:Lord's Day:worship ``f20!}.\footnote{\ John 20:19\index{43 John:ch.  20, v.  19, pg. ``f30!}.} Moreover, the Lord sanctified\index{Sabbath:Lord's Day:sanctified by the Lord ``f20!} the entire Lord's Day to himself and in it gives his people a foretaste of their eternal enjoyment of him and his glorious Kingdom\index{Kingdom of God/Christ ``f20!}.\footnote{\ Exodus 20:8\index{02 Exodus:ch.  20, v.   8, pg. ``f30!}; Hebrews 4:9\index{58 Hebrews:ch.   4, v.   9, pg. ``f30!}; Hebrews 10:25\index{58 Hebrews:ch.  10, v.  25, pg. ``f30!}.} Therefore, the congregation should ordinarily assemble for public worship\index{Worship:public:Lord's Day ``f20!} in the morning and evening on the Lord's Day. The specific times and location\index{Church courts:administrative procedures ``f20!}s of each service\index{Worship:service:times and locations ``f20!}, as a circumstance of worship\index{Worship:circumstances ``f20!}, are left to the discretion\index{Session:discretionary items ``f20!} and wisdom of the session.
\end{innerlst}

\subsection{The Elements and Ordinances of Public Worship} \index{Worship:public:ordinances ``f20!}\index{Worship:public:Lord's Day ``f20!}

\begin{innerlst}[resume*]
      \item The elements\index{Worship:elements:elements ``f20!} and ordinances\index{Worship:ordinances ``f20!} of New Testament\index{Worship:elements:instituted:New Testament ``f20!} public worship\index{Worship:public:Lord's Day ``f20!} prescribed and\index{Worship:elements:instituted:in Scripture ``f20!} instituted in the Scriptures are: a call to worship\index{Worship:public:call ``f20!},\footnote{\ Genesis 4:26\index{01 Genesis:ch.   4, v.  26, pg. ``f30!}; 1 Corinthians 1:2\index{46 1 Corinthians:ch.   1, v.   2, pg. ``f30!}; 2 Timothy 2:2\index{55 2 Timothy:ch.   2, v.   2, pg. ``f30!}2\index{55 2 Timothy:ch.   2, v.  22, pg. ``f30!}.} prayer,\footnote{\ 1 Timothy 2:1\index{54 1 Timothy:ch.   2, v.   1, pg. ``f30!}; Philippians 4:6\index{50 Philippians:ch.   4, v.   6, pg. ``f30!}.} reading the Word\index{Worship:public:scripture reading ``f20!},\footnote{\ Acts 15:2\index{44 Acts:ch.  15, v.   2, pg. ``f30!}1\index{44 Acts:ch.  15, v.  21, pg. ``f30!}; Colossians 4:16\index{51 Colossians:ch.   4, v.  16, pg. ``f30!}; 1 Timothy 4:1\index{54 1 Timothy:ch.   4, v.   1, pg. ``f30!}3\index{54 1 Timothy:ch.   4, v.  13, pg. ``f30!}.} preaching the Word\index{Preaching:God's Word ``f20!},\footnote{\ 1 Timothy 4:1\index{54 1 Timothy:ch.   4, v.   1, pg. ``f30!}3\index{54 1 Timothy:ch.   4, v.  13, pg. ``f30!}; 1 Timothy 5:1\index{54 1 Timothy:ch.   5, v.   1, pg. ``f30!}7\index{54 1 Timothy:ch.   5, v.  17, pg. ``f30!}; 2 Timothy 4:2\index{55 2 Timothy:ch.   4, v.   2, pg. ``f30!}.} hearing\index{Worship:hearing God's Word read and preached ``f20!} the Word, congregational unaccompanied\index{Worship:congregational singing of Psalms ``f20!} singing\index{Psalms:singing:book of Psalms ``f20!} of Psalms,\footnote{\ 2 Chronicles 29:30\index{14 2 Chronicles:ch.  29, v.  30, pg. ``f30!}; Matthew 26:30\index{40 Matthew:ch.  26, v.  30, pg. ``f30!}; 1 Corinthians 14:26\index{46 1 Corinthians:ch.  14, v.  26, pg. ``f30!}; Ephesians 5:19\index{49 Ephesians:ch.   5, v.  19, pg. ``f30!}; Colossians 3:16\index{51 Colossians:ch.   3, v.  16, pg. ``f30!}.} observing the sacraments\index{Worship:public:observation of sacraments ``f20!},\footnote{\ 1 Corinthians 11:33-34\index{46 1 Corinthians:ch.  11, v.  33-34, pg. ``f30!}.} giving of tithes\index{Worship:public:tithes:tithes ``f20!} and offerings\index{Worship:public:freewill offerings ``f20!},\footnote{\ Genesis 14:18-20\index{01 Genesis:ch.  14, v.  18-20, pg. ``f30!}; Psalm 96:8\index{19 Psalms:ch.  96, v.   8, pg. ``f30!}; 1 Corinthians 16:2\index{46 1 Corinthians:ch.  16, v.   2, pg. ``f30!}.} the giving and taking of oaths\index{Worship:oaths and vows ``f20!} and vows, and a benediction\index{Worship:benediction:benediction ``f20!}.\footnote{\ Numbers 6:23-27\index{04 Numbers:ch.   6, v.  23-27, pg. ``f30!}; 2 Corinthians 13:1\index{47 2 Corinthians:ch.  13, v.   1, pg. ``f30!}4\index{47 2 Corinthians:ch.  13, v.  14, pg. ``f30!}.}
      \item All elements\index{Worship:elements:elements ``f20!} and ordinances\index{Worship:ordinances ``f20!} are to be observed in\index{Worship:regulated by the Bible ``f20!} a manner consistent with Christ's prescription as revealed in the Scriptures\index{Bible:Word of God:standard:for the Church ``f20!}, being a clear application of our Lord's command\index{Lord's Supper:administered according to:Lord's command ``f20!}, ``Teaching\index{Church:teach all that Jesus commanded ``f20!} them to observe all that I have commanded you.''\footnote{\ Matthew 28:20\index{40 Matthew:ch.  28, v.  20, pg. ``f30!}; 1 Corinthians 4:6\index{46 1 Corinthians:ch.   4, v.   6, pg. ``f30!}; 1 Corinthians 11:2\index{46 1 Corinthians:ch.  11, v.   2, pg. ``f30!}.}
\end{innerlst}

\paragraph[Order of Worship and Leadership]{Order of Worship and Leadership} \index{Worship:public:liturgy:order of service ``f20!}

\begin{innerlst}[resume*]
      \item A fixed order of worship\index{Worship:public:liturgy:order of service ``f20!} is not prescribed in\index{Worship:public:liturgy:not prescribed in Scripture ``f20!} the Scriptures. Nevertheless, having an order of service is a common and accepted practice.\footnote{\ 1 Corinthians 14:26\index{46 1 Corinthians:ch.  14, v.  26, pg. ``f30!}, 33, 40\index{46 1 Corinthians:ch.  14, v.  26, 33, 40, pg. ``f30!}.} However, care ought to be taken that the elements\index{Worship:elements:elements ``f20!} are observed in\index{Worship:regulated by the Bible ``f20!} a manner that is suitable to the capacity of those\index{Worship:public:geared to capacity of worshippers ``f20!} worshipping. The following is a suggested order of worship:
      \begin{itemize}[noitemsep]
        \item Call to\index{Worship:public:call ``f20!} Worship and Greeting
        \item Prayer
        \item Singing\index{Psalms:singing:book of Psalms ``f20!} of a Psalm
        \item Scripture Reading\index{Worship:Scripture reading ``f20!} -- this may be a Psalm which is then briefly expounded\index{Preaching:exposition of:Scripture or doctrine ``f20!}
        \item Singing\index{Psalms:singing:book of Psalms ``f20!} of a Psalm
        \item Baptism, when\index{Baptism:during public worship service ``f20!} occasioned
        \item Prayer
        \item Singing\index{Psalms:singing:book of Psalms ``f20!} of a Psalm
        \item Scripture Reading\index{Worship:Scripture reading ``f20!}
        \item Sermon
        \item Prayer
        \item The Lord's Supper\index{Lord's Supper:Lord's Supper ``f20!}, when celebrated\index{Lord's Supper:celebration ``f20!}
        \item Singing\index{Psalms:singing:book of Psalms ``f20!} of a Psalm
        \item Collection of Tithes\index{Worship:public:tithes:collection ``f20!} and Offerings\index{Worship:public:freewill offerings ``f20!}
        \item Prayer  \item Benediction\index{Worship:benediction:benediction ``f20!}
      \end{itemize}
   \item Pastors\index{Pastor:pastor ``f20!} and Teachers\index{Teacher:teachers ``f20!}, as ministers\index{Pastor:minister of the Word ``f20!} of the Word, shall ordinarily conduct public worship\index{Worship:public:Lord's Day ``f20!} and administer the sacraments\index{Sacraments:administration:only by ordained pastor/teacher ``f20!}.\footnote{\ Colossians 1:17\index{51 Colossians:ch.   1, v.  17, pg. ``f30!}; 1 Timothy 5:1\index{54 1 Timothy:ch.   5, v.   1, pg. ``f30!}7\index{54 1 Timothy:ch.   5, v.  17, pg. ``f30!}; Galatians 6:6\index{48 Galatians:ch.   6, v.   6, pg. ``f30!}; Romans 12:8\index{45 Romans:ch.  12, v.   8, pg. ``f30!}; Hebrews 13:7\index{58 Hebrews:ch.  13, v.   7, pg. ``f30!}.}
   \item Ruling\index{Elders:ruling ``f20!} elders\index{Elders:elders ``f20!}, as ordained under-shepherds\index{Elders:under-shepherds ``f20!}, may call the congregation to worship\index{Worship:public:call ``f20!} and give the greeting, lead the congregation in prayer,\footnote{\ Romans 12:8\index{45 Romans:ch.  12, v.   8, pg. ``f30!}; 1 Timothy 5:1\index{54 1 Timothy:ch.   5, v.   1, pg. ``f30!}7\index{54 1 Timothy:ch.   5, v.  17, pg. ``f30!}; Hebrews 13:17\index{58 Hebrews:ch.  13, v.  17, pg. ``f30!}; 1 Peter 5:1\index{60 1 Peter:ch.   5, v.   1, pg. ``f30!}{-3}\index{60 1 Peter:ch.   5, v.   1-3, pg. ``f30!}.} introduce the Psalms to be sung, read the Scriptures\index{Elders:read the Scriptures in worship ``f20!}, and on occasion preach\index{Elders:preach on occasion ``f20!} the Word and pronounce the benediction\index{Worship:benediction:benediction ``f20!}.
   \item When the presbytery has granted permission, men training\index{Pastor:training ``f20!} for the Gospel ministry\index{Pastor:Gospel ministry ``f20!} may lead the congregation in public worship\index{Worship:public:Lord's Day ``f20!}. They are not permitted to administer the sacraments or\index{Sacraments:administration:only by ordained pastor/teacher ``f20!} pronounce the benediction\index{Worship:benediction:benediction ``f20!}.
   \item The session may occasionally request a lawfully\index{Pulpit:supply ``f20!} ordained pastor from another church to conduct the public worship\index{Worship:public:Lord's Day ``f20!}. However, if such supply is necessary for an extended period, permission must be granted by the presbytery.
\end{innerlst}

\paragraph[General Points About Public Worship]{General Points About Public Worship} \index{Worship:public:Lord's Day ``f20!} 

\begin{innerlst}[resume*]
      \item Those gathering for\index{Worship:gathering with joy and reverence ``f20!} worship are to be encouraged to do so with a joyful\index{Worship:with joyful, reverent heart ``f20!}, reverent heart. No persons should absent\index{Worship:public:absence ``f20!} themselves from the public worship\index{Worship:public:Lord's Day ``f20!} of God unless they are engaged in a work of necessity\index{Sabbath:works of necessity or mercy ``f20!} or mercy.\footnote{\ Exodus 19:8-11\index{02 Exodus:ch.  19, v.   8-11, pg. ``f30!}; Luke 13:10-17\index{42 Luke:ch.  13, v.  10-17, pg. ``f30!}.} Everyone must take care to be on time and if they are unable to be present when worship begins, they should join the assembly quietly\index{Worship:public:late persons enter assembly quietly ``f20!}.
      \item Once worship\index{Worship:public:duties of congregants ``f20!} has begun, the congregants shall give their full attention and refrain from any behaviour that might be unnecessarily distracting to others.
      \item It is commendable for infants and young\index{Worship:public:infants and young children present ``f20!} children to be in the public worship\index{Worship:public:Lord's Day ``f20!} services.\footnote{\ Exodus 10:9\index{02 Exodus:ch.  10, v.   9, pg. ``f30!}; Exodus 12:26\index{02 Exodus:ch.  12, v.  26, pg. ``f30!}; 2 Chronicles 20:13\index{14 2 Chronicles:ch.  20, v.  13, pg. ``f30!}; Matthew 19:14\index{40 Matthew:ch.  19, v.  14, pg. ``f30!}.} However, if a child becomes unsettled, a place should be provided for parents to\index{Church building:place for children if taken from service ``f20!} take their child. 
\end{innerlst} 

\paragraph[The Element of Public Prayer]{The Element of Public Prayer} \index{Prayer:public ``f20!} 

\begin{innerlst}[resume*]
      \item The officer\index{Office bearers:office-bearers ``f20!} leading in\index{Prayer:understanding of officer leading ``f20!} public prayer must have a solemn understanding\index{Prayer:solemn understanding of God's majesty ``f20!} of the majesty of God and a deep sense of his own sinful unworthiness\index{Sinful:unworthiness ``f20!}. He shall seek the help of the\index{Prayer:seek help of Holy Spirit ``f20!} Holy Spirit\index{Holy Spirit ``f20!} and pray with\index{Prayer:with confidence ``f20!} confidence that God will hear and answer according to his perfect will.
      \item The officer\index{Office bearers:office-bearers ``f20!} should think about the content of his prayers\index{Prayer:grounded in Word of God ``f20!} prior to worship\index{Worship:preparation for ``f20!}, and they should be grounded in God's Word, with\index{Prayer:to be grounded in the Bible ``f20!} special attention being given to the form which the Lord Jesus used in what is commonly called `The Lord's Prayer\index{Prayer:Lord's Prayer ``f20!}{'.}\footnote{\ See \textit{Larger Catechism}\index{Westminster:Larger Catechism ``f20!}, 178-196, and the original \textit{Westminster Directory of Worship} for helpful guides.}
      \item The prayer of `adoration\index{Prayer:adoration ``f20!} and invocation\index{Prayer:invocation ``f20!}{' should include adoration and thanksgiving, and}\index{Prayer:thanksgiving ``f20!} seek the presence, help\index{Prayer:seek help of Holy Spirit ``f20!}, and power of the Holy Spirit\index{Holy Spirit ``f20!} in the worship\index{Worship:pray for blessing upon ``f20!} of God.
      \item The prayer of `intercession\index{Prayer:intercessory ``f20!}{' should include confession, along with requests for such things as the needs of the local}\index{Prayer:needs for needs of local congregation ``f20!} congregation and community, the RP Church of Canada, and her global\index{Church:global sister churches ``f20!} sister churches; for the kingdom\index{Kingdom of God/Christ ``f20!} of Christ in her many branches; for our nation\index{Nations:Church universal ``f20!}, the nations of the world, and those who are in places of authority and trust.\footnote{\ Matthew 6:7-13\index{40 Matthew:ch.   6, v.   7-13, pg. ``f30!}; Mark 11:24\index{41 Mark:ch.  11, v.  24, pg. ``f30!}; Luke 18:1\index{42 Luke:ch.  18, v.   1, pg. ``f30!}; Ephesians 6:18, 19\index{49 Ephesians:ch.   6, v.  18, 19, pg. ``f30!}; 1 Thessalonians 5:16-18\index{52 1 Thessalonians:ch.   5, v.  16-18, pg. ``f30!}; 1 Timothy 2:1\index{54 1 Timothy:ch.   2, v.   1, pg. ``f30!}, 2\index{54 1 Timothy:ch.   2, v.   1, 2, pg. ``f30!}.} Prayers\index{Prayer:for preaching of God's Word ``f20!} for the preaching of God's Word\index{Preaching:God's Word ``f20!} and in relation to the sacraments should\index{Sacraments:sanctified by prayer ``f20!} seek the blessing of God upon the means of grace\index{Grace of God:means of ``f20!} and that the worship\index{Worship:pray for blessing upon ``f20!} offered would be acceptable.\footnote{\ Ephesians 6:18, 19\index{49 Ephesians:ch.   6, v.  18, 19, pg. ``f30!}; Psalm 15:1-5\index{19 Psalms:ch.  15, v.   1-5, pg. ``f30!}; Isaiah 29:13\index{23 Isaiah:ch.  29, v.  13, pg. ``f30!}; 1 Corinthians 11:2\index{46 1 Corinthians:ch.  11, v.   2, pg. ``f30!}8\index{46 1 Corinthians:ch.  11, v.  28, pg. ``f30!}.}
      \item The prayer of `thanksgiving\index{Prayer:thanksgiving ``f20!}{' should acknowledge the blessing and privilege of hearing}\index{Worship:hearing God's Word read and preached ``f20!} God's Word preached and seek the help of the\index{Prayer:seek help of Holy Spirit ``f20!} Holy Spirit\index{Holy Spirit ``f20!} to bless it.\footnote{\ Romans 10:15\index{45 Romans:ch.  10, v.  15, pg. ``f30!}; Titus 3:5\index{56 Titus:ch.   3, v.   5, pg. ``f30!}.}
      \item The Scriptures do\index{Prayer:posture:none explicitly commanded in Scripture ``f20!} not explicitly command one particular posture in prayer. The\index{Congregation:joining:prayer ``f20!}y do give the examples of kneeling\index{Prayer:posture:kneeling ``f20!}, bowing\index{Prayer:posture:bowing ``f20!}, sitting, standing, and\index{Prayer:posture:standing ``f20!} lying\index{Prayer:posture:lying on face ``f20!} on one's face as postures that are signs\index{Prayer:posture:signs of reverence ``f20!} of reverence and devotion.\footnote{\ Genesis 17:3, 7\index{01 Genesis:ch.  17, v.   3, 7, pg. ``f30!}; Genesis 24:52\index{01 Genesis:ch.  24, v.  52, pg. ``f30!}; 1 Kings 8:54\index{11 1 Kings:ch.   8, v.  54, pg. ``f30!}; Psalm 95:6\index{19 Psalms:ch.  95, v.   6, pg. ``f30!}; Ephesians 1:1\index{49 Ephesians:ch.   1, v.   1, pg. ``f30!}5--23\index{49 Ephesians:ch.   1, v.  15--23, pg. ``f30!}; Ephesians 2:4--7\index{49 Ephesians:ch.   2, v.   4--7, pg. ``f30!}.} 
\end{innerlst} 

\paragraph[The Element of the Public Reading of Holy Scripture]{The Element of the Public Reading of Holy Scripture} \index{Worship:Scripture reading ``f20!}\index{Bible:Word of God:holy ``f20!} 

\begin{innerlst}[resume*]
      \item All the canonical\index{Worship:public:scripture reading ``f20!} books of the Old and\index{Bible:Old and New Testaments ``f20!} New Testaments are to be publicly read, clearly and distinctly, using the best available common language translations\index{Bible:translations ``f20!} of the Bible. Paraphrased versions shall\index{Worship:public:no paraphrased versions of Psalms ``f20!} not be used as an element\index{Worship:public:elements ``f20!} of public worship\index{Worship:public:Lord's Day ``f20!}.\footnote{\ 1 Timothy 4:1\index{54 1 Timothy:ch.   4, v.   1, pg. ``f30!}3\index{54 1 Timothy:ch.   4, v.  13, pg. ``f30!}.} 
      \item The length of\index{Worship:public:scripture reading ``f20!} the passages to be read is left to the wisdom and discretion of the pastor. It is commendable to read a consecutive passage from each Testament in the services. While a goal should be to read through the whole of Scripture in the course of time, it is wise to read Scriptures which will best edify all present, for example, the Law, the Psalms, the Sermon on the Mount, etc. 
\end{innerlst} 

\paragraph[The Element of the Congregational Singing of Psalms]{The Element of the Congregational Singing of Psalms} \index{Psalms:singing:book of Psalms ``f20!} 

\begin{innerlst}[resume*]
      \item Congregational praise is\index{Worship:praise ``f20!} an element\index{Worship:public:elements ``f20!} of public worship\index{Worship:public:Lord's Day ``f20!} that is to be done according to the appointment and example of Jesus Christ and his apostles as revealed in God's Word. Therefore\index{Bible:Word of God:standard:of truth ``f20!}, only the ``psalms\index{Psalms:singing:book of Psalms ``f20!} and hymns\index{Psalms:biblical hymns ``f20!} and spiritual songs\index{Psalms:biblical songs ``f20!},'' from the book of\index{Worship:congregational singing of Psalms ``f20!} Psalms in the Bible, shall be sung in worship in accordance with what God commands in the Scriptures.\footnote{\ Matthew 26:30\index{40 Matthew:ch.  26, v.  30, pg. ``f30!}; Mark 14:26\index{41 Mark:ch.  14, v.  26, pg. ``f30!}; 1 Corinthians 14:26\index{46 1 Corinthians:ch.  14, v.  26, pg. ``f30!}; Ephesians 5:19\index{49 Ephesians:ch.   5, v.  19, pg. ``f30!}; Colossians 3:16\index{51 Colossians:ch.   3, v.  16, pg. ``f30!}.}  
      \item The Psalms, because of their divine inspiration\index{Psalms:divinely inspired ``f20!} and inherent\index{Psalms:inherent excellence ``f20!} excellence, are to be sung with heartfelt passion and thanksgiving by\index{Psalms:singing:with thanksgiving ``f20!} the congregation, to honour and praise\index{Worship:praise ``f20!} God. No instrumental\index{Worship:public:no instrumental accompaniment ``f20!} accompaniment is to be used. Precentors\index{Psalms:singing:precentor leading ``f20!} may lead the congregation in its praise. 
      \item The practice of giving a brief exposition\index{Preaching:exposition of:Scripture or doctrine ``f20!} of a Psalm (or portion thereof), during worship\index{Worship:exposition of a Psalm ``f20!} is commended, so that the congregation may become familiar with, and sing\index{Psalms:singing:book of Psalms ``f20!} through, the entire book of\index{Worship:congregational singing of Psalms ``f20!} Psalms. 
      \item To facilitate congregational\index{Worship:congregational singing of Psalms ``f20!} singing\index{Psalms:singing:book of Psalms ``f20!}, singable versions of\index{Psalms:translations:singable ``f20!} the Psalms are to be used. But like all translations\index{Bible:translations ``f20!} of Scripture, this is a circumstance of public worship\index{Worship:public:Lord's Day ``f20!}, and therefore no specific version is required. Nevertheless care should be taken that the text faithfully\index{Psalms:translations:faithful ``f20!} translates the original language and that the tune used for singing is appropriate. 
\end{innerlst} 

\paragraph[The Element of the Preaching the Word of God]{The Element of the Preaching the Word of God} \index{Preaching:God's Word ``f20!} 

\begin{innerlst}[resume*]
      \item The preaching of God's Word\index{Preaching:God's Word ``f20!}, being ``the power of God for salvation\index{Salvation:salvation ``f20!},'' should be done in such a way that the workman need not be ashamed but may save both himself and those who hear him.\footnote{\ Romans 1:16, 17\index{45 Romans:ch.   1, v.  16, 17, pg. ``f30!}; Romans 10:14-17\index{45 Romans:ch.  10, v.  14-17, pg. ``f30!}; 2 Timothy 2:15\index{55 2 Timothy:ch.   2, v.  15, pg. ``f30!}.}  
      \item The preacher\index{Preaching:preparation:careful study of Scriptures ``f20!} shall prepare his sermon\index{Preaching:sermon ``f20!} with careful study of the Scriptures\index{Pastor:careful study of the Scriptures ``f20!} and prayer, seeking\index{Preaching:preparation:prayer ``f20!} constantly the help\index{Preaching:preparation:help of Holy Spirit ``f20!} of the Holy Spirit\index{Holy Spirit ``f20!}, as he strives to declare faithfully the whole counsel\index{Preaching:whole counsel of God ``f20!} of God.\footnote{\ Acts 10:44\index{44 Acts:ch.  10, v.  44, pg. ``f30!}; Acts 20:27\index{44 Acts:ch.  20, v.  27, pg. ``f30!}; Philippians 1:1\index{50 Philippians:ch.   1, v.   1, pg. ``f30!}{-11}\index{50 Philippians:ch.   1, v.   1-11, pg. ``f30!}.} He should seek to read and keep abreast of the scholarship\index{Pastor:keep abreast of scholarship ``f20!} and of the social and doctrinal issues of the times.\footnote{\ 1 Chronicles 12:32\index{13 1 Chronicles:ch.  12, v.  32, pg. ``f30!}.}  
      \item While a pastor's personality will affect the structure and delivery of his sermons\index{Preaching:sermon ``f20!}, as a servant of\index{Pastor:servant of Christ ``f20!} Christ he is to perform his ministry according\index{Pastor:Gospel ministry ``f20!} to the following principles: 
      \begin{enumerate}[label=\alph*)]
            \item He is to preach\index{Preaching:attributes ``f20!} conscientiously\index{conscience ``f20!} and not negligently, labouring\index{Pastor:preach and teach ``f20!} in the work of preaching and\index{Preaching:God's Word ``f20!} teaching.\footnote{\ 1 Timothy 5:1\index{54 1 Timothy:ch.   5, v.   1, pg. ``f30!}6\index{54 1 Timothy:ch.   5, v.  16, pg. ``f30!}; 2 Timothy 2:15\index{55 2 Timothy:ch.   2, v.  15, pg. ``f30!}.} The sermon\index{Preaching:sermon ``f20!} is an exposition\index{Preaching:exposition of:Scripture or doctrine ``f20!} and application of Scripture. The continuous exposition of a book, or\index{Preaching:exposition of:a book of the Bible ``f20!} a section of a book, is a commendable approach. However, topical messages\index{Preaching:topical messages ``f20!} are also appropriate to address the needs of the congregation and issues of the nation\index{National:issues ``f20!}. These should be chosen wisely so that souls might\index{Preaching:draw souls to Christ ``f20!} be drawn to Christ and be conformed to his image.\footnote{\ Romans 8:29\index{45 Romans:ch.   8, v.  29, pg. ``f30!}.} 
            \item He is to preach\index{Preaching:attributes ``f20!} plainly, so that everyone may understand; delivering the truth not in the enticing words of man's wisdom, but in demonstration of the Spirit\index{Holy Spirit ``f20!} and of power, lest the cross of Christ be emptied of its power.\footnote{\ 1 Corinthians 2:1-5\index{46 1 Corinthians:ch.   2, v.   1-5, pg. ``f30!}.} He ought to use or quote ecclesiastical or\index{Preaching:discernment in quoting human writings ``f20!} other human writers with discernment and care. 
            \item He is to preach\index{Preaching:attributes ``f20!} faithfully, looking\index{Preaching:faithfully ``f20!} to the honour of Christ, the conversion\index{Preaching:looking for conversions ``f20!}, edification, and\index{Preaching:for edification of hearers ``f20!} salvation\index{Salvation:salvation ``f20!} of the people, and not to his own gain or glory. He\index{Preaching:not for preacher's own glory ``f20!} should keep nothing back which may promote these holy\index{Preaching:holy ends ``f20!} ends, without favouritism\index{Pastor:not show favouritism ``f20!}.\footnote{\ Acts 20:20\index{44 Acts:ch.  20, v.  20, pg. ``f30!}.} 
            \item He is to preach\index{Preaching:attributes ``f20!} wisely, framing all his doctrines\index{Preaching:care proclaiming doctrines ``f20!}, exhortations, comforts, and especially his reproofs, in a winsome\index{Preaching:winsome manner ``f20!} manner, with a view to the edification of the\index{Preaching:for edification of hearers ``f20!} hearers.\footnote{\ 2 Timothy 3:16\index{55 2 Timothy:ch.   3, v.  16, pg. ``f30!}; 1 Corinthians 9:22\index{46 1 Corinthians:ch.   9, v.  22, pg. ``f30!}.} 
            \item He is to preach\index{Preaching:attributes ``f20!} earnestly, as is proper to the Word\index{Bible:Word of God:standard:of truth ``f20!} of God, from the heart and to the heart.\footnote{\ Acts 20:26\index{44 Acts:ch.  20, v.  26, pg. ``f30!}.} 
            \item He is to preach\index{Preaching:attributes ``f20!} perceptively, taking into account the knowledge\index{Preaching:consider knowledge and understanding of congregation ``f20!} and understanding of the congregation.\footnote{\ Acts 17:16-34\index{44 Acts:ch.  17, v.  16-34, pg. ``f30!}.} 
            \item He is to preach\index{Preaching:attributes ``f20!} with loving affection, that the people may see that all he says comes from his godly\index{Preaching:godly zeal ``f20!} zeal and hearty desire to do them good.\footnote{\ Acts 20:31, 32\index{44 Acts:ch.  20, v.  31, 32, pg. ``f30!}.} 
            \item He is to preach\index{Preaching:attributes ``f20!} as taught of\index{Preaching:taught of God ``f20!} God, persuaded in his own heart that all he teaches is the truth and that as a pastor, he will be judged with greater strictness\index{Pastor:judged with greater strictness ``f20!} as one who must give an account.\footnote{\ 1 Corinthians 11:2\index{46 1 Corinthians:ch.  11, v.   2, pg. ``f30!}3\index{46 1 Corinthians:ch.  11, v.  23, pg. ``f30!}; Ephesians 6:19\index{49 Ephesians:ch.   6, v.  19, pg. ``f30!}; Hebrews 13:17\index{58 Hebrews:ch.  13, v.  17, pg. ``f30!}.} 
      \end{enumerate} 
      \item The congregation participates in the preaching of the\index{Preaching:God's Word ``f20!} Word as they listen with diligence, preparation, and prayer; receiving\index{Preaching:congregation should pray for ``f20!} it with faith and love; laying\index{Preaching:to be received by congregation ``f20!} it up in their hearts; and practicing it in their lives.\footnote{\ Acts 17:11\index{44 Acts:ch.  17, v.  11, pg. ``f30!}; James 1:19-25\index{59 James:ch.   1, v.  19-25, pg. ``f30!}.} 
\end{innerlst} 

\paragraph[The Element of the Tithes and Offerings]{The Element of the Tithes and Offerings} \index{Worship:public:tithes:collection ``f20!}\index{Worship:public:freewill offerings ``f20!}  

\begin{innerlst}[resume*]
      \item The collection of the tithes\index{Worship:public:tithes:tithes ``f20!} and freewill\index{Worship:public:freewill offerings ``f20!} offerings, being an element\index{Worship:public:elements ``f20!} of public worship\index{Worship:public:Lord's Day ``f20!}, should be taken up during public worship. Sessions have the discretion\index{Session:discretionary items ``f20!} to permit the use of electronic means prior to or after worship\index{Worship:public:tithes:electronic means ``f20!}.\footnote{\ Genesis 14:20\index{01 Genesis:ch.  14, v.  20, pg. ``f30!}; Psalm 96:8\index{19 Psalms:ch.  96, v.   8, pg. ``f30!}; 1 Corinthians 16:1-2\index{46 1 Corinthians:ch.  16, v.   1-2, pg. ``f30!}; 2 Corinthians 8:1\index{47 2 Corinthians:ch.   8, v.   1, pg. ``f30!}{-9:15}\index{47 2 Corinthians:ch.   8, v.   1-9:15, pg. ``f30!}.} The collection of \textit{weekly} Lord's Day tithes and freewill offerings is an element of worship.\footnote{\ Acts 5:2\index{44 Acts:ch.   5, v.   2, pg. ``f30!}.} 
      \item Special offerings\index{Worship:public:freewill offerings ``f20!} for specific needs may also be collected. 
\end{innerlst} 

\paragraph[The Element of the Benediction]{The Element of the Benediction} \index{Worship:benediction:benediction ``f20!} 

\begin{innerlst}[resume*]
      \item The benediction\index{Worship:benediction:benediction ``f20!} is to be understood as a pronouncement of God's blessing upon his people at the conclusion of the worship\index{Worship:benediction:blessing ``f20!} service. Words of benediction taken from Scripture are to be used. The Old Testament\index{Worship:benediction:example ``f20!} Trinitarian\index{Worship:benediction:Trinitarian ``f20!}, high priestly benediction, ``The LORD bless you and keep you; the LORD make his face to shine upon you and be gracious to you; the LORD lift\index{Worship:benediction:God's face ``f20!} up his countenance upon you and give you peace,'' or the Trinitarian, apostolic benediction, ``The grace\index{Worship:benediction:God's grace ``f20!} of the Lord Jesus Christ and the love of God and the fellowship of the Holy Spirit\index{Holy Spirit ``f20!} be with you all,'' are distinctly appropriate.\footnote{\ Numbers 6:24-26\index{04 Numbers:ch.   6, v.  24-26, pg. ``f30!}; 2 Corinthians 13:1\index{47 2 Corinthians:ch.  13, v.   1, pg. ``f30!}4\index{47 2 Corinthians:ch.  13, v.  14, pg. ``f30!}.} If, however, the pastor deems another benediction taken from Scripture more fitting for a particular occasion, he may use it. 
\end{innerlst} 

\subsection{The Ordinances of Baptism and the Lord's Supper} \index{Sacraments:holy ordinance ``f20!}\index{Sacraments:baptism and Lord's Supper ``f20!}\index{Lord's Supper:Lord's Supper ``f20!} 

\begin{innerlst}[resume*]
      \item A sacrament\index{Sacraments:holy ordinance ``f20!} is a holy ordinance instituted by Christ\index{Sacraments:instituted by Christ ``f20!}, in which symbols and\index{Sacraments:symbols ``f20!} actions signify Christ and the benefits of the covenant\index{Covenant:of grace:benefits ``f20!} of grace.\footnote{\ Matthew 26:26-30\index{40 Matthew:ch.  26, v.  26-30, pg. ``f30!}; Mark 14:22-26\index{41 Mark:ch.  14, v.  22-26, pg. ``f30!}; Luke 22:15-20\index{42 Luke:ch.  22, v.  15-20, pg. ``f30!}; 1 Corinthians 11:2\index{46 1 Corinthians:ch.  11, v.   2, pg. ``f30!}3\index{46 1 Corinthians:ch.  11, v.  23, pg. ``f30!}{-25}\index{46 1 Corinthians:ch.  11, v.  23-25, pg. ``f30!}.} Sacraments\index{Sacraments:means of grace ``f20!} become means of grace\index{Grace of God:means of ``f20!} and seals of the\index{Sacraments:seals of the covenant ``f20!} benefits of the covenant only by the blessing of Christ and the working of his Spirit\index{Holy Spirit ``f20!} in those who by faith receive them. 
      \item There are two sacraments\index{Sacraments:two ``f20!} of the New Covenant\index{Covenant:new ``f20!} instituted by Christ\index{Sacraments:instituted by Christ ``f20!}, namely, Baptism and\index{Sacraments:baptism and Lord's Supper ``f20!} the Lord's Supper\index{Lord's Supper:Lord's Supper ``f20!}. They are to be administered\index{Sacraments:administration:according to Christ's appointment ``f20!} according to Christ's appointment, by pastors\index{Sacraments:administration:only by ordained pastor/teacher ``f20!}/teachers\index{Teacher:teachers ``f20!} at a time when the congregation assembles for public worship\index{Worship:public:Lord's Day ``f20!}.\footnote{\ 1 Corinthians 11:2\index{46 1 Corinthians:ch.  11, v.   2, pg. ``f30!}3\index{46 1 Corinthians:ch.  11, v.  23, pg. ``f30!}{-25}\index{46 1 Corinthians:ch.  11, v.  23-25, pg. ``f30!}.} In the case of those unable to attend public worship, the sacraments may\index{Sacraments:administration:to those unable to attend public worship ``f20!} be administered apart from a regular service\index{Sacraments:administration:apart from regular service ``f20!}. In such cases, the congregation must be represented by other members in addition to the pastor/teacher, and there should be a brief ministry of the Word\index{Pastor:minister of the Word ``f20!} of God. 
\end{innerlst} 

\subsection{The Ordinance of Baptism} \index{Baptism:ordinance ``f20!} 

\begin{innerlst}[resume*]
      \item Baptism is only to be administered\index{Baptism:administration:by ordained pastor/teacher ``f20!} by an ordained pastor\index{Sacraments:administration:only by ordained pastor/teacher ``f20!}/teacher\index{Teacher:teachers ``f20!} of Christ. 
\end{innerlst} 

\paragraph[The Baptism of an Adult]{The Baptism of an Adult} \index{Baptism:covenant:child ``f20!}\index{Baptism:adult ``f20!} 

\begin{innerlst}[resume*]
      \item An adult\index{Baptism:adult ``f20!} is to be baptized upon\index{Baptism:credible profession of faith ``f20!} a credible profession of faith\index{profession of faith ``f20!} provided that he/she has not already received a valid Christian\index{Baptism:valid Christian ``f20!} baptism.\footnotemark{. }\footnotetext{\ See, \textit{Westminster Confession}\index{Westminster:Confession of Faith ``f20!}\textit{ of Faith}, 28-29.}  
      \item Before baptism is administered, instruction\index{Baptism:give instruction about nature and purpose ``f20!} should be given as to the institution\index{Sacraments:explain institution to congregation ``f20!}, nature, and purpose of the sacrament.\index{Baptism:sacrament ``f20!} Suitable\index{Worship:Scripture reading ``f20!} Scripture should be chosen for the occasion and instruction\index{Sacraments:give instruction about nature and purpose ``f20!}.\footnote{\ E.g., Matthew 28:18-20\index{40 Matthew:ch.  28, v.  18-20, pg. ``f30!}; Ezekiel 36:25-27\index{26 Ezekiel:ch.  36, v.  25-27, pg. ``f30!}.} The following (or similar) instruction shall either be explicit in the sermon\index{Preaching:sermon ``f20!} or given in the form of a statement: 
\end{innerlst}
\textit{{``Baptism is a}}\index{Baptism:sacrament ``f20!}\textit{ sacrament ordained by our}\index{Baptism:ordained by Jesus ``f20!}\textit{ Lord Jesus Christ. It is a sign and seal of the inclusion}\index{Baptism:sign and seal of the covenant ``f20!}\textit{ of the person who is baptized in the}\index{Baptism:covenant:of grace:under ``f20!}\textit{ covenant of grace. Baptism with water}\index{Baptism:water:teaches ``f20!}\textit{ teaches that we and our children}\index{Children:conceived and born in sin ``f20!}\textit{ are conceived and born in sin}\index{Sin:born in ``f20!}\textit{. It signifies our}\index{Baptism:signifies and seals:dying to sin ``f20!}\textit{ dying to sin}\index{Sin:dying to ``f20!}\textit{ and our rising to newness of life by virtue of our union}\index{Church:union with Christ ``f20!}\textit{ with Christ in his death}\index{Jesus:death and resurrection ``f20!}\textit{ and resurrection}\index{Jesus:resurrection ``f20!}\textit{. It also signifies and seals to}\index{Baptism:signifies and seals:cleansing from sin ``f20!}\textit{ us cleansing}\index{Baptism:sign of cleansing from sin ``f20!}\textit{ from sin}\index{Sin:cleansing signified and sealed in baptism ``f20!}\textit{ by the blood and Spirit}\index{Holy Spirit ``f20!}\textit{ of Christ. Since these gifts}\index{Salvation:gifts ``f20!}\textit{ of salvation}\index{Salvation:salvation ``f20!}\textit{ are the gracious provision of the}\index{Baptism:seal of cleansing ``f20!}\textit{ triune}\index{Baptism:in name of Trinity ``f20!}\textit{ God, who is pleased to claim us as his very own, we are baptized in the name of the Father and of the Son}\index{Jesus:Son ``f20!}\textit{ and of the Holy Spirit. Baptized persons}\index{Baptism:covenant:of grace:obligations ``f20!}\textit{ are called upon to assume the obligations of the covenant}\index{Covenant:of grace:obligations ``f20!}\textit{ of grace; baptism summons}\index{Baptism:summons us to renounce sin ``f20!}\textit{ us to renounce sin}\index{Sin:admonishment to:renounce ``f20!}\textit{ and the world, and to walk }\textit{humbly}\index{Christian:duty:walk:humbly ``f20!}\textit{ with our God in devotion to his commandments.''}   

\begin{innerlst}[resume*]
      \item The person to be baptized shall\index{Baptism:credible profession of faith ``f20!} give public profession of faith\index{profession of faith ``f20!} by means of assent to the \textit{Covenant of Church}\index{Covenant of Church Membership ``f20!}\textit{ Membership} (see, Official Vows\index{vows, official ``f20!} 1). 
      \item The pastor shall ask the congregation to rise and respond either by verbal assent or\index{Congregation:voting procedures:assent ``f20!} by raising the right hand to the following vow\index{Baptism:vows:congregation ``f20!}: 
\end{innerlst}
\textit{{``Do you, the members}}\index{Baptism:vows:congregation ``f20!}\textit{ of this congregation, receive this person into your fellowship}\index{Baptism:welcome into fellowship of congregation ``f20!}\textit{ and promise to pray}\index{Congregation:promise to pray for baptized persons ``f20!}\textit{ for him/her, and to help and encourage him/her in the Christian life?''}   

\begin{innerlst}[resume*]
      \item The pastor shall lead in prayer, thanking\index{Prayer:thanksgiving ``f20!} God for his grace\index{Prayer:thank God for his grace ``f20!} and seeking his blessing on the ordinance of\index{Baptism:ordinance ``f20!} baptism, and\index{Baptism:prayer for blessing ``f20!} then baptize the\index{Baptism:administration:procedures ``f20!} person, stating the person's name and saying, \textit{{``(Name), I baptize you}}\index{Baptism:in name of Trinity ``f20!}\textit{ in the name of the Father, and of the Son}\index{Jesus:Son ``f20!}\textit{, and of the Holy Spirit}\index{Holy Spirit ``f20!}\textit{, one God, blessed forever, Amen.''} 
      \item A member of the session may conclude with prayer\index{Session:meetings:prayer ``f20!}, that the grace\index{Grace of God:signified and sealed in baptism ``f20!} signified and\index{Baptism:signifies and seals:grace ``f20!} sealed in\index{Baptism:grace signified and sealed ``f20!} baptism would\index{Baptism:covenant:of grace:signed and sealed ``f20!} be abundantly realized in the person's life. 
\end{innerlst} 

\paragraph[The Baptism of a Covenant Child]{The Baptism of a Covenant Child} \index{Baptism:covenant:child ``f20!} 

\begin{innerlst}[resume*]
      \item Parents\index{Baptism:parents to receive instruction ``f20!}, or legal guardians\index{Baptism:guardians in place of parents ``f20!}, of a covenant\index{Baptism:covenant:child ``f20!} child, having received instruction on\index{Baptism:give instruction about nature and purpose ``f20!} the nature and purpose of baptism, shall\index{Baptism:during public worship service ``f20!} have their child baptized during a public worship\index{Worship:public:Lord's Day ``f20!} service.  
      \item Whether baptism is by\index{Baptism:modes ``f20!} immersion, pouring, or sprinkling; whether a parent\index{Baptism:parents' involvement ``f20!} or pastor holds the\index{Baptism:administration:procedures ``f20!} infant\index{Baptism:covenant:child ``f20!} child during the baptism; whether one or three applications of water\index{Baptism:water:medium ``f20!} are given when God's triune\index{Baptism:in name of Trinity ``f20!} name is pronounced, etc., are circumstantial and therefore are left to the discretion\index{Session:discretionary items ``f20!} of the session. 
      \item During the service\index{Baptism:give instruction about nature and purpose ``f20!}, instruction should be given to the congregation as to the institution\index{Sacraments:explain institution to congregation ``f20!}, nature, and purpose of the sacrament.\index{Baptism:sacrament ``f20!} Suitable\index{Worship:Scripture reading ``f20!} Scripture should be chosen for the occasion and instruction\index{Sacraments:give instruction about nature and purpose ``f20!}. The following (or similar) instruction shall either be explicit in the sermon\index{Preaching:sermon ``f20!} or given in the form of a statement: 
\end{innerlst}
\textit{{``Baptism is a}}\index{Baptism:sacrament ``f20!}\textit{ sacrament ordained by our}\index{Baptism:ordained by Jesus ``f20!}\textit{ Lord Jesus Christ. It is a sign and seal of the inclusion}\index{Baptism:sign and seal of the covenant ``f20!}\textit{ of the person who is baptized in the}\index{Baptism:covenant:of grace:under ``f20!}\textit{ covenant of grace. Baptism with water}\index{Baptism:water:teaches ``f20!}\textit{ teaches that we and our children}\index{Children:conceived and born in sin ``f20!}\textit{ are conceived and born in sin}\index{Sin:born in ``f20!}\textit{. It signifies our}\index{Baptism:signifies and seals:dying to sin ``f20!}\textit{ dying to sin}\index{Sin:dying to ``f20!}\textit{ and our rising to newness of life by virtue of our union}\index{Church:union with Christ ``f20!}\textit{ with Christ in his death}\index{Jesus:death and resurrection ``f20!}\textit{ and resurrection}\index{Jesus:resurrection ``f20!}\textit{. It also signifies and seals to}\index{Baptism:signifies and seals:cleansing from sin ``f20!}\textit{ us cleansing}\index{Baptism:sign of cleansing from sin ``f20!}\textit{ from sin}\index{Sin:cleansing signified and sealed in baptism ``f20!}\textit{ by the blood and Spirit}\index{Holy Spirit ``f20!}\textit{ of Christ. Since these gifts}\index{Salvation:gifts ``f20!}\textit{ of salvation}\index{Salvation:salvation ``f20!}\textit{ are the gracious provision of the}\index{Baptism:seal of cleansing ``f20!}\textit{ triune}\index{Baptism:in name of Trinity ``f20!}\textit{ God, who is pleased to claim us as his very own, we are baptized in the name of the Father and of the Son}\index{Jesus:Son ``f20!}\textit{ and of the Holy Spirit. Baptized persons}\index{Baptism:covenant:of grace:obligations ``f20!}\textit{ are called upon to assume the obligations of }\textit{the covenant}\index{Covenant:of grace:obligations ``f20!}\textit{. Baptism summons}\index{Baptism:summons us to renounce sin ``f20!}\textit{ us to renounce sin}\index{Sin:admonishment to:renounce ``f20!}\textit{ and the world, and to walk humbly}\index{Christian:duty:walk:humbly ``f20!}\textit{ with our God in devotion to his commandments.''}  \textit{{``Although our young}}\index{Baptism:covenant:child ``f20!}\textit{ children do}\index{Children:limited understanding, yet to be baptized ``f20!}\textit{ not yet understand these things, they are nevertheless to be baptized. For the promise of the covenant is made to believers and}\index{Covenant:believers and their children ``f20!}\textit{ to their children, as}\index{Children:covenant:under ``f20!}\textit{ God declared to Abraham}\index{Covenant:Abraham ``f20!}\textit{, `And I will establish My covenant between Me and you and your descendants after you in their generations, for an everlasting covenant}\index{Covenant:everlasting ``f20!}\textit{, to be God to you and your descendants after you'.}\footnote{\ Genesis 17:7\index{01 Genesis:ch.  17, v.   7, pg. ``f30!}.}\textit{ Under the New Covenant}\index{Covenant:new ``f20!}\textit{, no less than in the Old Covenant}\index{Covenant:old ``f20!}\textit{, the children of believers, have, by virtue of their birth, an interest in the covenant and a right to the seal of it}\index{Baptism:sign and seal of the covenant ``f20!}\textit{. The covenant}\index{Covenant:of grace:OT and NT ``f20!}\textit{ of grace is the same in substance in both the Old and}\index{Covenant:Old and New Testaments ``f20!}\textit{ the New Covenants, and baptism has replaced circumcision}\index{Baptism:replaced circumcision ``f20!}\textit{ as the seal of that covenant}\index{Baptism:covenant:seal ``f20!}\textit{.}\footnote{\ Colossians 2:11-12\index{51 Colossians:ch.   2, v.  11-12, pg. ``f30!}.}\textit{ Our Saviour}\index{Jesus:Saviour ``f20!}\textit{ admitted little}\index{Baptism:Jesus admitted children to his presence ``f20!}\textit{ children into}\index{Children:Jesus welcomed ``f20!}\textit{ his presence, embracing them and blessing them, and saying, `Of such is the kingdom}\index{Kingdom of God/Christ ``f20!}\textit{ of God'.}\footnote{\ Mark 10:14\index{41 Mark:ch.  10, v.  14, pg. ``f30!}.}\textit{ The grace}\index{Worship:benediction:God's grace ``f20!}\textit{ signified in}\index{Baptism:signifies and seals:grace ``f20!}\textit{ baptism is not tied to the moment of administration}\index{Baptism:administration:timing of application of grace ``f20!}\textit{. Scripture teaches that our children}\index{Children:covenant:holy ``f20!}\textit{ are covenantally holy before}\index{Covenant:children holy before baptism ``f20!}\textit{ their baptism.}\footnote{\ 1 Corinthians 7:14\index{46 1 Corinthians:ch.   7, v.  14, pg. ``f30!}.}\textit{ Baptism applies}\index{Baptism:covenant:of grace:promises ``f20!}\textit{ the promises and obligations of the covenant}\index{Covenant:of grace:obligations ``f20!}\textit{ to our children}\index{Children:covenant:obligations ``f20!}\textit{ and calls them to personal repentance}\index{repentance ``f20!}\textit{ and faith as they come to years of understanding}\index{Baptism:years of understanding ``f20!}\textit{.''}   

\begin{innerlst}[resume*]
      \item After the instruction\index{Baptism:administration:procedures ``f20!}, the parent\index{Baptism:parents' involvement ``f20!}(s)/legal guardian\index{Baptism:guardians in place of parents ``f20!}(s) are to bring the child to the front of the congregation. An unbelieving\index{Baptism:unbelieving parent/guardian accompanying ``f20!} parent/legal guardian may be invited to accompany the believing parent/legal guardian in presenting the child. However, the unbelieving parent/legal guardian should not be asked to give assent to the Covenant of Baptism\index{Baptism:covenant:assent ``f20!}. 
      \item The pastor shall ask the parent\index{Baptism:parents' involvement ``f20!}(s)/legal guardian\index{Baptism:guardians in place of parents ``f20!}(s) to respond to the following question:  
\end{innerlst}
\textit{{``Do you publicly}}\index{Church membership:vows:public ``f20!}\textit{ renew your profession of faith}\index{profession of faith ``f20!}\textit{ in Christ as expressed in your vows of church}\index{Church membership:vows:renewal ``f20!}\textit{ membership?''}   

\begin{innerlst}[resume*]
      \item The believing parent\index{Baptism:parents' involvement ``f20!}(s)/legal guardian\index{Baptism:guardians in place of parents ``f20!}(s) shall then assent to the \textit{Parental}\index{Baptism:vows:parent ``f20!}\textit{/Legal Guardian Baptismal Vows} in relation to their child (see, Official Vows\index{vows, official ``f20!}). 
      \item The pastor should ask the congregation to rise and respond to the following vow\index{Baptism:vows:congregation ``f20!}: 
\end{innerlst}
\textit{{``Do you, the members}}\index{Baptism:vows:congregation ``f20!}\textit{ of this congregation, receive this child into your fellowship}\index{Baptism:welcome into fellowship of congregation ``f20!}\textit{ and promise to pray}\index{Congregation:promise to pray for baptized persons ``f20!}\textit{ for him/her, and to help and encourage the parent}\index{Baptism:parents' involvement ``f20!}\textit{s as they seek to bring him/her up in the nurture and admonition of}\index{Parents:nurture and admonish children in the Lord ``f20!}\textit{ the Lord?''}   

\begin{innerlst}[resume*]
      \item The pastor shall lead in prayer, thanking\index{Prayer:thanksgiving ``f20!} God for his grace\index{Prayer:thank God for his grace ``f20!} and seeking his blessing upon the ordinance of\index{Baptism:ordinance ``f20!} baptism. 
      \item The pastor shall then baptize the\index{Baptism:administration:procedures ``f20!} child, stating the child's name and saying, 
\end{innerlst}
\textit{{``(Name), I baptize you}}\index{Baptism:in name of Trinity ``f20!}\textit{ in the name of the Father, and of the Son}\index{Jesus:Son ``f20!}\textit{, and of the Holy Spirit}\index{Holy Spirit ``f20!}\textit{, one God, blessed forever, Amen.''}   

\begin{innerlst}[resume*]
      \item A member of the session may conclude with prayer\index{Session:meetings:prayer ``f20!}, that the grace\index{Grace of God:signified and sealed in baptism ``f20!} signified and\index{Baptism:signifies and seals:grace ``f20!} sealed in\index{Baptism:grace signified and sealed ``f20!} baptism would\index{Baptism:covenant:of grace:signed and sealed ``f20!} be abundantly realized in the child's life. 
      \item An accurate record should be kept in\index{Baptism:records to be kept ``f20!} the session minutes of all\index{Session:minutes:baptisms ``f20!} persons baptized, with\index{Baptism:administration:procedures ``f20!} the date, and, in the case of a child, the names of the parent\index{Baptism:parents' involvement ``f20!}(s)/legal guardian\index{Baptism:guardians in place of parents ``f20!}(s), and the child's date of birth. A Certificate of Baptism may\index{Baptism:certificate ``f20!} be provided for each person baptized. 
\end{innerlst}

\subsection{The Ordinance of the Lord's Supper} \index{Lord's Supper:Lord's Supper ``f20!}

\begin{innerlst}[resume*]
      \item The Lord's Supper\index{Lord's Supper:Lord's Supper ``f20!} is to be celebrated frequently\index{Lord's Supper:celebrate frequently ``f20!} at the discretion\index{Session:discretionary items ``f20!} of the session. Only an ordained pastor\index{Sacraments:administration:only by ordained pastor/teacher ``f20!} may administer the sacrament.\footnote{\ Acts 2:42\index{44 Acts:ch.   2, v.  42, pg. ``f30!}; Acts 20:7\index{44 Acts:ch.  20, v.   7, pg. ``f30!}, 11\index{44 Acts:ch.  20, v.   7, 11, pg. ``f30!}; 1 Corinthians 11:2\index{46 1 Corinthians:ch.  11, v.   2, pg. ``f30!}, 23, 26\index{46 1 Corinthians:ch.  11, v.   2, 23, 26, pg. ``f30!}.} 
      \item Whether believers are\index{Lord's Supper:procedures ``f20!} to take the bread\index{Lord's Supper:bread ``f20!} into their hands and divide among themselves, or eat what is given to them; whether they are to move to be seated at tables; whether elders\index{Elders:elders ``f20!} or deacons\index{Deacons:deacons ``f20!} may help in distributing the elements\index{Lord's Supper:elements:distribution ``f20!}; whether the communicant\index{Church membership:communicant ``f20!} is to return the cup\index{Lord's Supper:cups ``f20!} or bread to the deacon or elder or hand it to another communicant member; whether the bread is to be leavened\index{Lord's Supper:bread, leavened or unleavened ``f20!} or unleavened, and the wine to be red or white\index{Lord's Supper:wine:red or white ``f20!}, fermented\index{Lord's Supper:wine:fermented or unfermented ``f20!} or unfermented; whether one cup is to be shared by all, or many cups derived and distributed from the one cup, etc., are not matters essential to the validity of the ordinance and\index{Lord's Supper:ordinance ``f20!} are at the discretion\index{Session:discretionary items ``f20!} of the session. 
      \item The visible\index{Church:visible ``f20!} Church is described in the Bible as both a living body, and\index{Church:body of Christ ``f20!} a kingdom\index{Kingdom of God/Christ ``f20!}, to which keys\index{Church:keys of the kingdom ``f20!} are given.\footnote{\ Matthew 16:19\index{40 Matthew:ch.  16, v.  19, pg. ``f30!}; Matthew 18:18\index{40 Matthew:ch.  18, v.  18, pg. ``f30!}.} The session ought to make efforts to ensure that the Lord's Supper\index{Lord's Supper:Lord's Supper ``f20!} is only given to those who are baptized and\index{Lord's Supper:for baptized members:in good standing ``f20!} professing members\index{Church membership:professing members ``f20!} in good standing\index{Church membership:good standing ``f20!} in a congregation of the visible church.\footnotemark{.}\footnotetext{\ See, \textit{Westminster Confession}\index{Westminster:Confession of Faith ``f20!}\textit{ of Faith}, 25.} \footnote{\ 1 Corinthians 11:2\index{46 1 Corinthians:ch.  11, v.   2, pg. ``f30!}8\index{46 1 Corinthians:ch.  11, v.  28, pg. ``f30!}; 1 Corinthians 5:9-13\index{46 1 Corinthians:ch.   5, v.   9-13, pg. ``f30!}.} 
      \item The pastor should give instruction\index{Lord's Supper:give instruction about nature and purpose ``f20!} as to the institution\index{Sacraments:explain institution to congregation ``f20!}, nature, and purpose of the Lord's Supper\index{Lord's Supper:Lord's Supper ``f20!}, drawing attention to the words of institution in the\index{Lord's Supper:institution ``f20!} Gospels and 1 Corinthians. The following (or similar) instruction shall be explicit in the sermon\index{Preaching:sermon ``f20!} or given in the form of a statement: 
\end{innerlst}
\textit{{``The Lord's Supper}}\index{Lord's Supper:Lord's Supper ``f20!}\textit{ is an ordinance}\index{Lord's Supper:ordinance ``f20!}\textit{ instituted by our}\index{Lord's Supper:instituted by Jesus ``f20!}\textit{ Lord Jesus Christ. It is to be observed until}\index{Lord's Supper:observed until Jesus returns ``f20!}\textit{ he comes again, in remembrance}\index{Lord's Supper:remembrance of Jesus sacrifice ``f20!}\textit{ of the sacrifice of himself which he offered upon the cross. The physical elements}\index{Lord's Supper:elements:symbols of Christ's body ``f20!}\textit{ of bread}\index{Lord's Supper:bread ``f20!}\textit{ and wine represent the body and}\index{Lord's Supper:Jesus' body and blood ``f20!}\textit{ blood of the Saviour}\index{Jesus:Saviour ``f20!}\textit{ and are received by true believers as}\index{Lord's Supper:for believers ``f20!}\textit{ signs and seals of all}\index{Lord's Supper:signifies and seals:benefits of Christ's sacrifice ``f20!}\textit{ the benefits of his sacrifice. The }\textit{Supper signifies and seals remission}\index{Lord's Supper:signifies and seals:remission from sin ``f20!}\textit{ of sins}\index{Sin:remission of ``f20!}\textit{, and nourishes our souls to}\index{Lord's Supper:draw souls to Christ ``f20!}\textit{ grow in Christ, and is a bond and pledge of}\index{Lord's Supper:pledge of union with Christ ``f20!}\textit{ our union}\index{Church:union with Christ ``f20!}\textit{ and communion with him and with each other as members of his body, the}\index{Church:body of Christ ``f20!}\textit{ Church. It assures us that God is faithful to fulfill the promises of}\index{Lord's Supper:covenant:promise ``f20!}\textit{ the Covenant of Grace}\index{Covenant:of grace:promises ``f20!}\textit{, and it calls us to renewed commitment to obey and serve the Lord in}\index{Lord's Supper:renewed commitment to the Lord in gratitude ``f20!}\textit{ gratitude for}\index{Lord's Supper:thanksgiving ``f20!}\textit{ his salvation}\index{Salvation:salvation ``f20!}\textit{. Christ himself is present by his Spirit}\index{Holy Spirit ``f20!}\textit{ in the Supper, to make it truly a means of grace}\index{Grace of God:means of ``f20!}\textit{ to those who receive it in faith. Those who partake}\index{Lord's Supper:participation of communicants ``f20!}\textit{ of the Supper do so in thankful remembrance that the body of Christ was given, and his blood shed, for them. They rejoice in hope as they anticipate the completion of their redemption in that day when they will share in the marriage supper}\index{Jesus:marriage supper ``f20!}\textit{ of the Lamb}\index{Marriage:supper of the Lamb ``f20!}\textit{.''}  

\begin{innerlst}[resume*]
      \item The pastor shall then draw attention to the words of warning and\index{Lord's Supper:warning ``f20!} invitation\index{Lord's Supper:invitation from Jesus ``f20!} found in 1 Corinthians 11:2\index{46 1 Corinthians:ch.  11, v.   2, pg. ``f30!}7-34\index{46 1 Corinthians:ch.  11, v.  27-34, pg. ``f30!}. Such warning may be in the following (or similar) words: 
\end{innerlst}
\textit{{``It is the duty of}}\index{Lord's Supper:warning ``f20!}\textit{ the Church to warn you that if you do not trust in the Lord}\index{Church discipline:call for trust in Lord Jesus Christ ``f20!}\textit{ Jesus Christ for your salvation}\index{Salvation:salvation ``f20!}\textit{, or if you are living an ungodly, disobedient life, and have not repented, you should not partake}\index{Lord's Supper:participation of communicants ``f20!}\textit{ of the Lord's Supper}\index{Lord's Supper:Lord's Supper ``f20!}\textit{, lest you eat and drink condemnation}\index{Lord's Supper:eat and drink condemnation ``f20!}\textit{ to yourself. The}\index{Lord's Supper:self-examination ``f20!}\textit{ Lord's Supper is for repentant}\index{repentance ``f20!}\textit{ and believing sinners, who, after}\index{Sin:confession of ``f20!}\textit{ examining themselves and seeking reconciliation}\index{reconciliation ``f20!}\textit{ with their brothers and sisters, come confessing Christ as their Saviour}\index{Jesus:Saviour ``f20!}\textit{.''}  \textit{{``This warning is}}\index{Lord's Supper:warning ``f20!}\textit{ not designed to keep the humble}\index{humility ``f20!}\textit{ and contrite away from the Lord's Supper}\index{Lord's Supper:Lord's Supper ``f20!}\textit{. On the contrary, the Supper is a means of grace}\index{Grace of God:means of ``f20!}\textit{ offered to sustain weak}\index{Lord's Supper:sustain:weak pilgrims ``f20!}\textit{ pilgrims on their journey}\index{Lord's Supper:sustain:pilgrims on their journey ``f20!}\textit{ through the wilderness of this life. We who come to partake}\index{Lord's Supper:participation of communicants ``f20!}\textit{ of the symbols of}\index{Lord's Supper:elements:symbols of Christ's body ``f20!}\textit{ Christ's body and}\index{Lord's Supper:Jesus' body and blood ``f20!}\textit{ blood, come as sinners whose}\index{Sinful:unworthiness ``f20!}\textit{ only hope is the grace}\index{Grace of God:salvation depends on ``f20!}\textit{ of God in Christ. We come in a worthy manner}\index{Lord's Supper:worthy manner of participation ``f20!}\textit{ if we recognize that in ourselves, we are unworthy sinners who need a Saviour}\index{Jesus:Saviour ``f20!}\textit{, if we discern his body given for our sins}\index{Sin:remission of ``f20!}\textit{, and if we hunger and thirst after Christ, giving thanks}\index{Lord's Supper:giving thanks ``f20!}\textit{ for his grace}\index{Prayer:thank God for his grace ``f20!}\textit{, trusting in his merits, feeding on him by faith, and renewing our covenant}\index{Lord's Supper:covenant:renewal ``f20!}\textit{ with him and his people.''}  \textit{{``If you are prepared to come in this way, then hear the Lord's words}}\index{Lord's Supper:administered according to:Lord's example ``f20!}\textit{ of gracious invitation}\index{Lord's Supper:invitation from Jesus ``f20!}\textit{.''}   

\begin{innerlst}[resume*]
      \item The pastor shall take the bread\index{Lord's Supper:bread ``f20!} and the cup\index{Lord's Supper:cups ``f20!}, and exhibit\index{Lord's Supper:elements:exhibit as sacramental symbols ``f20!} them to the communicants\index{Lord's Supper:participation of communicants ``f20!}, using words such as these: 
\end{innerlst}
\textit{{``The Lord Jesus, the}}\index{Lord's Supper:administered according to:Lord's example ``f20!}\textit{ same night in which he was betrayed, took bread}\index{Lord's Supper:bread ``f20!}\textit{ and also the cup}\index{Lord's Supper:cups ``f20!}\textit{. Following his example, and ministering in his name, I take this bread and this cup, and exhibit}\index{Lord's Supper:elements:exhibit as sacramental symbols ``f20!}\textit{ them to you as the sacramental symbols of}\index{Lord's Supper:elements:symbols of Christ's body ``f20!}\textit{ the body and}\index{Lord's Supper:Jesus' body and blood ``f20!}\textit{ blood of the Lord.''}   

\begin{innerlst}[resume*]
      \item Replacing the elements\index{Lord's Supper:elements:consecrated by prayer ``f20!}, he should say the following (or similar) words: 
\end{innerlst}
\textit{{``After the Lord Jesus had}}\index{Lord's Supper:administered according to:Lord's example ``f20!}\textit{ taken the bread}\index{Lord's Supper:bread ``f20!}\textit{ and the cup}\index{Lord's Supper:cups ``f20!}\textit{, he blessed them. Let us pray, as we}\index{Lord's Supper:prayer ``f20!}\textit{ give thanks, and consecrate these elements}\index{Lord's Supper:elements:consecrated by prayer ``f20!}\textit{.''}     

\begin{innerlst}[resume*]
      \item A prayer should\index{Lord's Supper:prayer of praise ``f20!} be offered to praise\index{Prayer:praise ``f20!} God for his grace\index{Prayer:thank God for his grace ``f20!} in bringing salvation\index{Salvation:salvation ``f20!}; reaffirm the trust of God's people in God's grace\index{Prayer:trust in God's grace ``f20!} and Christ's righteousness and\index{Righteousness:Jesus' ``f20!} mediation\index{Jesus:mediation ``f20!}; and plead for the Lord to grant\index{Prayer:ask Lord to grant working the Holy Spirit ``f20!} the gracious, effectual working of his Spirit\index{Holy Spirit ``f20!} through the sacrament\index{Lord's Supper:sacrament ``f20!}.\footnote{\ 1 Corinthians 1:4\index{46 1 Corinthians:ch.   1, v.   4, pg. ``f30!}; Philippians 1:3-11\index{50 Philippians:ch.   1, v.   3-11, pg. ``f30!} Colossians 1:3-6 1:28; 1 Corinthians 5:9-13\index{46 1 Corinthians:ch.   5, v.   9-13, pg. ``f30!}.} 
      \item The pastor shall take the bread\index{Lord's Supper:bread ``f20!} (or a portion of it), and break it, and say the following (or similar) words: 
\end{innerlst}
\textit{{``After the Lord Jesus had}}\index{Lord's Supper:administered according to:Lord's example ``f20!}\textit{ blessed the bread}\index{Lord's Supper:bread ``f20!}\textit{, he broke it. Following his command and example, and ministering in his name, I break this bread the bread is broken and give it to you his disciples, saying as he said, ``Take, eat; this is my body which}\index{Lord's Supper:Jesus' body and blood ``f20!}\textit{ is broken for you; do this in remembrance}\index{Lord's Supper:remembrance of Jesus sacrifice ``f20!}\textit{ of me.''}  

\begin{innerlst}[resume*]
      \item The bread\index{Lord's Supper:bread ``f20!} is then distributed to\index{Lord's Supper:elements:distribution ``f20!} the communicants\index{Lord's Supper:participation of communicants ``f20!}, including the elders\index{Elders:elders ``f20!}, who receive it and partake of it. During the distribution, appropriate Scriptures\index{Lord's Supper:read appropriate Scriptures ``f20!} may be read, or Psalms sung. 
      \item Then, the pastor shall take the cup\index{Lord's Supper:cups ``f20!} and offer it to the congregation, and say the following (or similar) words: 
\end{innerlst}
\textit{{`In the same manner he also took the cup}}\index{Lord's Supper:cups ``f20!}\textit{ after supper, saying, ``This cup is the new covenant}\index{Covenant:new ``f20!}\textit{ in my blood; this do, as often as you drink it, in remembrance}\index{Lord's Supper:remembrance of Jesus sacrifice ``f20!}\textit{ of me.'' For as often as you eat this bread}\index{Lord's Supper:bread ``f20!}\textit{ and drink this cup, you proclaim the Lord's death until}\index{Lord's Supper:Jesus' death proclaimed ``f20!}\textit{ he comes.'}  

\begin{innerlst}[resume*]
      \item The cup\index{Lord's Supper:cups ``f20!} is then distributed to\index{Lord's Supper:elements:distribution ``f20!} the communicants\index{Lord's Supper:participation of communicants ``f20!}, including the elders\index{Elders:elders ``f20!}, who receive it and partake of it. During the distribution, appropriate Scriptures\index{Lord's Supper:read appropriate Scriptures ``f20!} may be read, or Psalms may be sung. 
      \item After all the communicants\index{Lord's Supper:participation of communicants ``f20!} have partaken, a brief address may be given, emphasizing the grace\index{Grace of God:in Jesus ``f20!} of God in Jesus Christ as set forth in the sacrament\index{Lord's Supper:sacrament ``f20!}, and ``exhorting them to continue in the faith.''\footnote{\ Acts 14:22\index{44 Acts:ch.  14, v.  22, pg. ``f30!}.} 
\end{innerlst}

\item
\restartlist{innerlst}
\section{Chapter 2 -- Days of Fasting or Thanksgiving}  \index{Fasting:fasting ``f20!}\index{Prayer:thanksgiving ``f20!} 

\begin{innerlst}[resume*]
      \item Under the New Testament, the\index{Worship:public:morning and evening:New Testament ``f20!}re\index{Sabbath:Lord's Day:New Testament requirement ``f20!} is no day commanded in Scripture to be kept holy\index{Sabbath:to be kept holy ``f20!} but the Lord's Day, which is the Christian Sabbath\index{Sabbath:Sabbath ``f20!}. Nevertheless, it may be appropriate to separate a day or days for public fasting\index{Fasting:fasting ``f20!} or thanksgiving, as\index{Fasting:thanksgiving ``f20!} extraordinary dispensations of God's providence give\index{Fasting:extraordinary dispensations of providence ``f20!} occasion. 
\end{innerlst} 

\subsection{Fasting} \index{Fasting:fasting ``f20!} 

\begin{innerlst}[resume*]
      \item Special days of fasting\index{Fasting:fasting ``f20!}, humiliation\index{Fasting:humiliation ``f20!}, and prayer are\index{Fasting:accompanied with prayer ``f20!} particularly appropriate when God's judgments\index{National:judgment ``f20!} are evident in the land, or when corporate\index{Fasting:when corporate sin manifested ``f20!} sin in the Church or nation\index{National:sin provokes the Lord ``f20!} provokes the Lord and invites his judgments.  
      \item In Christian fasting\index{Fasting:fasting ``f20!}, the believer voluntarily abstains from food or some ordinary lawful\index{Fasting:abstain from lawful pleasures for a season ``f20!} pleasure for a season, for the purpose of seeking the will of God,\footnote{\ Judges 20:26\index{07 Judges:ch.  20, v.  26, pg. ``f30!}{-28}\index{07 Judges:ch.  20, v.  26-28, pg. ``f30!}; Acts 14:23\index{44 Acts:ch.  14, v.  23, pg. ``f30!}.} strengthening for service\index{Fasting:strengthening for service ``f20!}, spiritual growth, deliverance\index{Fasting:spiritual growth ``f20!} or personal protection,\footnote{\ 2 Chronicles 20:3\index{14 2 Chronicles:ch.  20, v.   3, pg. ``f30!}{-4}\index{14 2 Chronicles:ch.  20, v.   3-4, pg. ``f30!}; Ezra 8:21-23\index{15 Ezra:ch.   8, v.  21-23, pg. ``f30!}; Esther 4:6\index{17 Esther:ch.   4, v.   6, pg. ``f30!}.} overcoming temptation\index{temptation ``f20!}, expressing grief\index{Fasting:expressing grief for sins ``f20!},\footnote{Judges 20:26}\index{07 Judges:ch.  20, v.  26, pg. ``f30!}-28\index{07 Judges:ch.  20, v.  26-28, pg. ``f30!}\; 1 Samuel 20:34\index{09 1 Samuel:ch.  20, v.  34, pg. ``f30!}; 1 Samuel 31:13\index{09 1 Samuel:ch.  31, v.  13, pg. ``f30!}; 2 Samuel 1:11-12\index{10 2 Samuel:ch.   1, v.  11-12, pg. ``f30!}. and declaring love for God and worshiping\index{Fasting:worship ``f20!} him.\footnote{Luke 2:37\index{42 Luke:ch.   2, v.  37, pg. ``f30!}.} It should be accompanied\index{Fasting:accompanied with prayer ``f20!} by prayer,\footnote{\ Nehemiah 1:3, 4\index{16 Nehemiah:ch.   1, v.   3, 4, pg. ``f30!}; Daniel 9:13\index{27 Daniel:ch.   9, v.  13, pg. ``f30!}.} meditation\index{meditation on God's Word ``f20!}, self-examination\index{Fasting:self-examination ``f20!}, humiliation\index{Fasting:humiliation ``f20!}\footnote{\ 1 Kings 21:27-29\index{11 1 Kings:ch.  21, v.  27-29, pg. ``f30!}; Psalm 35:13\index{19 Psalms:ch.  35, v.  13, pg. ``f30!}.} before God, confession of sin\index{Sin:confession of ``f20!}, repentance\index{repentance ``f20!},\footnote{\ 1 Samuel 7:6\index{09 1 Samuel:ch.   7, v.   6, pg. ``f30!}; Joel 2:12\index{29 Joel:ch.   2, v.  12, pg. ``f30!}; Jonah 3:5-8\index{32 Jonah:ch.   3, v.   5-8, pg. ``f30!}.} and renewed dedication to a life of obedience\index{Fasting:renewed dedication to obedience ``f20!}.\footnote{\ Ezra 4:16\index{15 Ezra:ch.   4, v.  16, pg. ``f30!}; Ezra 8:23\index{15 Ezra:ch.   8, v.  23, pg. ``f30!}; Esther 4:16\index{17 Esther:ch.   4, v.  16, pg. ``f30!}; Daniel 1:12\index{27 Daniel:ch.   1, v.  12, pg. ``f30!}; Acts 9:9\index{44 Acts:ch.   9, v.   9, pg. ``f30!}.}  
      \item Fasts\index{Fasting:fasting ``f20!} may be partial or absolute.\footnote{\ Ezra 4:16\index{15 Ezra:ch.   4, v.  16, pg. ``f30!}; Esther 4:16\index{17 Esther:ch.   4, v.  16, pg. ``f30!}; Daniel 1:12\index{27 Daniel:ch.   1, v.  12, pg. ``f30!}; Acts 9:9\index{44 Acts:ch.   9, v.   9, pg. ``f30!}.} They may be private/family\index{Fasting:private/family ``f20!},\footnote{\ Matthew 6:16-18\index{40 Matthew:ch.   6, v.  16-18, pg. ``f30!}.} congregational,\footnote{\ Joel 2:15-16\index{29 Joel:ch.   2, v.  15-16, pg. ``f30!}; Acts 13:2\index{44 Acts:ch.  13, v.   2, pg. ``f30!}.} or nation\index{National:sin provokes the Lord ``f20!}al\index{National:fasts ``f20!}.\footnote{\ 2 Chronicles 20:3\index{14 2 Chronicles:ch.  20, v.   3, pg. ``f30!}; Nehemiah 9:1\index{16 Nehemiah:ch.   9, v.   1, pg. ``f30!}; Jonah 3:5-8\index{32 Jonah:ch.   3, v.   5-8, pg. ``f30!}.} They may be regular or\index{Fasting:regular or occasional ``f20!} occasional.\footnote{\ Leviticus 16:29-31; Matthew 9:15\index{40 Matthew:ch.   9, v.  15, pg. ``f30!}.} They may last for a part, or for the entirety of a day or longer.\footnote{\ Judges 20:26\index{07 Judges:ch.  20, v.  26, pg. ``f30!}; 1 Samuel 7:6\index{09 1 Samuel:ch.   7, v.   6, pg. ``f30!}; 2 Samuel 1:12\index{10 2 Samuel:ch.   1, v.  12, pg. ``f30!}; 2 Samuel 12:16-23\index{10 2 Samuel:ch.  12, v.  16-23, pg. ``f30!}; Daniel 10:3-13\index{27 Daniel:ch.  10, v.   3-13, pg. ``f30!}; Acts 27:33-34\index{44 Acts:ch.  27, v.  33-34, pg. ``f30!}; Luke 2:37\index{42 Luke:ch.   2, v.  37, pg. ``f30!}.} They must be undertaken with a view to meeting the needs of others.\footnote{\ Isaiah 58:1-14\index{23 Isaiah:ch.  58, v.   1-14, pg. ``f30!}.} 
      \item A fast\index{Fasting:fast day ``f20!} day may be marked\index{Fasting:may be marked as service of public worship ``f20!} by a service of public worship\index{Worship:public:Lord's Day ``f20!}. In such services, it is fitting that\index{Fasting:Psalms of penitence ``f20!} Psalms of penitence\index{Psalms:penitence ``f20!} be sung, along with the offering\index{Fasting:accompanied with prayer ``f20!} of prayers\index{Prayer:confession of sin ``f20!} of confession of sin\index{Sin:confession of ``f20!} and petitions for pardon.\footnote{\ Nehemiah 1:4\index{16 Nehemiah:ch.   1, v.   4, pg. ``f30!}; Daniel 9:3\index{27 Daniel:ch.   9, v.   3, pg. ``f30!}; Joel 2:12\index{29 Joel:ch.   2, v.  12, pg. ``f30!}; Acts 13:2\index{44 Acts:ch.  13, v.   2, pg. ``f30!}.} 
      \item If the civil government calls for a time of prayer and\index{Fasting:accompanied with prayer ``f20!} fasting\index{Fasting:fasting ``f20!} that is in harmony with\index{Fasting:called by civil government, if in harmony with Scripture ``f20!} the Scriptures, sessions should\index{Fasting:sessions should encourage ``f20!} encourage the people of God to pay due respect to that call.\footnote{\ 2 Chronicles 20:3\index{14 2 Chronicles:ch.  20, v.   3, pg. ``f30!}; Nehemiah 9:1\index{16 Nehemiah:ch.   9, v.   1, pg. ``f30!}; Jonah 3:5-8\index{32 Jonah:ch.   3, v.   5-8, pg. ``f30!}.}  
      \item Apart from such general occasions, there may be times when families\index{Fasting:private/family ``f20!} and individuals, for their own reasons, give themselves to prayer and\index{Fasting:accompanied with prayer ``f20!} fasting\index{Fasting:fasting ``f20!} for a season. 
\end{innerlst} 

\subsection{Thanksgiving} \index{Prayer:thanksgiving ``f20!} 

\begin{innerlst}[resume*]
      \item Blessed with the hope of salvation\index{Salvation:salvation ``f20!} in Christ,\footnote{\ John 3:16\index{43 John:ch.   3, v.  16, pg. ``f30!}; 1 Peter 1:1-9\index{60 1 Peter:ch.   1, v.   1-9, pg. ``f30!}; James 1:12\index{59 James:ch.   1, v.  12, pg. ``f30!}.} Christians should always be thankful.\footnote{\ 1 Chronicles 16:8-12\index{13 1 Chronicles:ch.  16, v.   8-12, pg. ``f30!}; Psalm 105:1\index{19 Psalms:ch. 105, v.   1, pg. ``f30!}; Psalm 136:26\index{19 Psalms:ch. 136, v.  26, pg. ``f30!}; Ephesians 5:20\index{49 Ephesians:ch.   5, v.  20, pg. ``f30!}; Philippians 4:4-7\index{50 Philippians:ch.   4, v.   4-7, pg. ``f30!}.} Nevertheless, there are occasions when special seasons of corporate\index{Worship:seasons of corporate thanksgiving ``f20!} thanksgiving should be observed\index{Congregation:thanksgiving ``f20!}.\footnote{\ Psalm 107:8, 9\index{19 Psalms:ch. 107, v.   8, 9, pg. ``f30!}.} These may be in response to a particular blessing of God in the life of the congregation, or to a call by the civil authority for a day of national thanksgiving because\index{National:thanksgiving ``f20!} of God's provision of protection or material blessings. On such occasions, suitable Psalms of thanksgiving should be used\index{Psalms:singing:with thanksgiving ``f20!}, and the preaching of God's Word\index{Preaching:God's Word ``f20!} should be on the theme of gratitude to God. 
\end{innerlst}

\item
\restartlist{innerlst}
\section{Chapter 3 -- Weddings and Funerals}  \index{Marriage:wedding ``f20!}\index{Funeral:funeral ``f20!}Weddings\index{Marriage:wedding ``f20!} and funerals\index{Funeral:funeral ``f20!} are solemn public occasions for which the following guidelines are suggested as a help to pastors.  
\subsection{Weddings} \index{Marriage:wedding ``f20!} 

\begin{innerlst}[resume*]
      \item Marriage is ordained\index{Marriage:ordained by God ``f20!} by God for the welfare\index{Marriage:for welfare of humanity ``f20!} and happiness of humanity. God has ordained that marriage is between\index{Marriage:between one man and one woman ``f20!} one man and one woman, for their joy and sanctification\index{sanctification ``f20!}, for the raising of\index{Marriage:for the raising of children ``f20!} children, and for the more certain continuance\index{Marriage:children for continuance of the Church ``f20!} of the Church. In marriage\index{Marriage:leave parents ``f20!}, husband and wife\index{wives ``f20!} leave their parents and cleave to one another faithfully and\index{Marriage:faithfulness ``f20!} are not separated except by death\index{Marriage:separation only by death ``f20!}.\footnote{Genesis 1:28\index{01 Genesis:ch.   1, v.  28, pg. ``f30!}; Genesis 2:23-25\index{01 Genesis:ch.   2, v.  23-25, pg. ``f30!}; John 2:1-12\index{43 John:ch.   2, v.   1-12, pg. ``f30!}.} 
      \item God instituted marriage\index{Marriage:instituted:by God ``f20!} at the beginning of time\index{Marriage:instituted:at beginning of time ``f20!}, and it is therefore neither a sacrament\index{Marriage:not a sacrament ``f20!} of, nor an ordinance\index{Marriage:not ordinance peculiar to the Church ``f20!} peculiar to, the Church. It is integral\index{Marriage:integral to all societies ``f20!} to all societies and nations and\index{Nations:marriage universal ``f20!} is therefore rightly recognized by both Church and State\index{Civil magistrates:State ``f20!}. The Church should respect and abide by all reasonable and sound civil regulations that\index{Civil magistrates:regulations ``f20!} do not violate Scripture. The pastor should ensure that sound State regulations are fulfilled\index{Marriage:state regulations to be fulfilled ``f20!}, while also keeping\index{Marriage:records to be kept ``f20!} the Church's own record of marriages.  
      \item A Christian may marry whom\index{Marriage:may marry whom wish, but only in Lord ``f20!} he/she wishes, but only in the Lord.\footnote{Genesis 24.1-67\index{01 Genesis:ch.  24, v.   1-67, pg. ``f30!}; Genesis 27:46-28\index{01 Genesis:ch.  27, v.  46-28, pg. ``f30!}; Proverbs 31:10-31\index{20 Proverbs:ch.  31, v.  10-31, pg. ``f30!}; 1 Corinthians 7:39\index{46 1 Corinthians:ch.   7, v.  39, pg. ``f30!}; 2 Corinthians 6:14\index{47 2 Corinthians:ch.   6, v.  14, pg. ``f30!}; Ephesians 5:21\index{49 Ephesians:ch.   5, v.  21, pg. ``f30!}{-23}\index{49 Ephesians:ch.   5, v.  21-23, pg. ``f30!}.} The Lord instructs\index{Marriage:instructed by the Lord to live harmoniously ``f20!} all husbands and wives\index{wives ``f20!} to live harmoniously together and has instituted marriage\index{Marriage:instituted:by God ``f20!} as an analogy\index{Marriage:analogy for Jesus and Church ``f20!} of the love between Jesus Christ and his Church.\footnote{Ephesians 5:21\index{49 Ephesians:ch.   5, v.  21, pg. ``f30!}{-23}\index{49 Ephesians:ch.   5, v.  21-23, pg. ``f30!}; Colossians 3:18, 19\index{51 Colossians:ch.   3, v.  18, 19, pg. ``f30!}.} 
      \item A notice\index{Marriage:advance notice given ``f20!} of the marriage will be given on the two Lord's Days prior\index{Congregation:meetings:notices ``f20!} to the marriage ceremony. 
      \item If the marriage is to be legally\index{Marriage:legal recognition ``f20!} recognized by the State\index{Civil magistrates:State ``f20!}, either a marriage licence\index{Marriage:licence ``f20!} must be obtained or the requirements for the publication of Banns\index{Marriage:by Banns ``f20!} will be undertaken according to local regulations. The\index{Civil magistrates:regulations ``f20!} form of the Banns may be as follows: 
\end{innerlst}
\textit{{``I hereby publish the Banns}}\index{Marriage:by Banns ``f20!}\textit{ of Marriage between}\index{Marriage:advance notice given ``f20!}\textit{ Miss N of [City, Province], and Mr. N. of [City, Province], who are engaged to be married on the nth day of [Month, year] in [City, Province]. If any of you know cause, or just impediment}\index{Marriage:just impediment to be declared ``f20!}\textit{, why these two persons should not be joined}\index{Marriage:joined together ``f20!}\textit{ together in holy}\index{Marriage:holy estate ``f20!}\textit{ Marriage, you are to declare it to the elders}\index{Elders:elders ``f20!}\textit{ of this congregation. This is the first [second, or third] time of publishing these Banns.''}   

\begin{innerlst}[resume*]
      \item Weddings\index{Marriage:wedding ``f20!} should be held on a day of the week other than on the Lord's Day\index{Sabbath:Lord's Day:worship ``f20!}.  
      \item In addition to prayer\index{Marriage:prayer:at wedding ``f20!}, singing\index{Psalms:singing:book of Psalms ``f20!}, Scripture reading, and\index{Worship:public:scripture reading ``f20!} preaching, a\index{Marriage:preaching at wedding ``f20!} wedding may include the following:  
      \begin{itemize}[noitemsep]
            \item Declaration of purpose. 
            \item Declaration of the bride's parent\index{Marriage:bride's:responsibilities ``f20!}(s). 
            \item Exchange of marriage vows\index{Marriage:vows:exchange ``f20!}. 
            \item Exchange of rings\index{Marriage:exchange of rings ``f20!}. 
      \end{itemize} 
      \item Declaration of purpose - The pastor may use words such as these: 
\end{innerlst}
\textit{{``We are gathered together here in the sight of God, and in the presence of these witnesses, to}}\index{Marriage:before witnesses ``f20!}\textit{ join together this man and this woman in holy}\index{Marriage:holy estate ``f20!}\textit{ matrimony, which is an honourable estate, instituted by God}\index{Marriage:instituted:by God ``f20!}\textit{, which signifies to}\index{Marriage:signifies mystical union with Christ ``f20!}\textit{ us the mystical union}\index{Church:union with Christ ``f20!}\textit{ of Jesus Christ and his Church. God has ordained that}\index{Marriage:ordained by God ``f20!}\textit{ marriage is between}\index{Marriage:between one man and one woman ``f20!}\textit{ one man and one woman, for their joy and sanctification}\index{sanctification ``f20!}\textit{, for the raising of}\index{Marriage:for the raising of children ``f20!}\textit{ children, and for the more certain continuance}\index{Marriage:children for continuance of the Church ``f20!}\textit{ of the Church. In marriage}\index{Marriage:leave parents ``f20!}\textit{ husband and wife}\index{wives ``f20!}\textit{ leave their parents and cleave to one another faithfully and}\index{Marriage:faithfulness ``f20!}\textit{ are not separated except by death}\index{Marriage:separation only by death ``f20!}\textit{. Jesus Christ honoured marriage by}\index{Marriage:Jesus honoured by his presence ``f20!}\textit{ his presence and by doing his first miraculous}\index{Jesus:first miraculous sign ``f20!}\textit{ sign at a wedding.}\index{Marriage:wedding ``f20!}\textit{ Furthermore, he declared, ``What God has joined}\index{Marriage:joined together ``f20!}\textit{ together, let no man separate}\index{Marriage:no man separate ``f20!}\textit{.'' Marriage is therefore}\index{Marriage:not enter into:unadvisedly ``f20!}\textit{ not to be entered into unadvisedly or lightly}\index{Marriage:not enter into:lightly ``f20!}\textit{, but reverently, discreetly, advisedly, soberly, and in the }\textit{fear of God. Into this holy estate [name of the Groom], and [name of the Bride], come now to be joined.}  \textit{Do you, or any person here today, know of any reason why you may not be legally}\index{Marriage:legal recognition ``f20!}\textit{ married? If you do, please speak now or forever hold your peace.''}  

\begin{innerlst}[resume*]
      \item Declaration of the bride's parent\index{Marriage:bride's:responsibilities ``f20!}(s) - The pastor may ask either the father or mother\index{Marriage:bride's:parents' declaration ``f20!} of the bride: 
\end{innerlst}
\textit{{``Do you give this woman to be married to this man?''}}  

\begin{innerlst}[resume*]
      \item Exchange of marriage vows\index{Marriage:vows:exchange ``f20!} - An example of vows that\index{Marriage:vows:example ``f20!} may be used (either in a repeat-after-me form or be responded to by the statement ``I do'' by both groom and bride)\index{Marriage:bride's:responsibilities ``f20!} is as follows: 
\end{innerlst}
{``}\textit{I [name of the Groom] take you [name of the Bride] to be my wedded}\index{Marriage:wedded spouse ``f20!}\textit{ wife}\index{wives ``f20!}\textit{ and do, in the presence of God and before these witnesses}\index{Marriage:before witnesses ``f20!}\textit{, promise and covenant, to follow}\index{Marriage:covenant ``f20!}\textit{ the example of Christ who has loved the Church and given himself for her, as I love, help}\index{Marriage:vows:love and help ``f20!}\textit{, and guide you}\index{Marriage:vows:blessing ``f20!}\textit{ as we live together in holiness by God's grace}\index{Grace of God:marriage depends on ``f20!}\textit{ until he shall separate}\index{Marriage:until death separates ``f20!}\textit{ us by death}\index{Marriage:separation only by death ``f20!}\textit{.'' }  \textit{{``I [name of the Bride] take you [name of the Groom] to be my wedded}}\index{Marriage:wedded spouse ``f20!}\textit{ husband and do in the presence of God and before these witnesses}\index{Marriage:before witnesses ``f20!}\textit{, promise and covenant, to follow}\index{Marriage:covenant ``f20!}\textit{ the commands of God to love, obey, and help}\index{Marriage:vows:obey and help ``f20!}\textit{ you as we live together in holiness by God's grace}\index{Grace of God:marriage depends on ``f20!}\textit{ until he shall separate}\index{Marriage:until death separates ``f20!}\textit{ us by death}\index{Marriage:separation only by death ``f20!}\textit{.}   

\begin{innerlst}[resume*]
      \item Exchange of rings\index{Marriage:exchange of rings ``f20!} - If rings are used, the pastor may ask: 
\end{innerlst}
\textit{{``What pledge do}}\index{Marriage:pledge to support vows ``f20!}\textit{ you give of your marriage vows}\index{Marriage:vows:exchange ``f20!}\textit{?''}  

\begin{innerlst}[resume*]
      \item As each ring is presented to its recipient, the pastor may say: 
\end{innerlst}
\textit{{``Give and receive this ring as a token of your marriage vows}}\index{Marriage:vows:exchange ``f20!}\textit{. May it be to you a symbol of the value, constancy, and purity of your wedded}\index{Marriage:wedded spouse ``f20!}\textit{ love, and a seal of the solemn}\index{Marriage:vows:ring as seal ``f20!}\textit{ vows you have made to one another before God.''}   

\begin{innerlst}[resume*]
      \item The pastor may then say: 
\end{innerlst}
\textit{{``By virtue of the authority vested in me as a pastor of the Gospel, and}}\index{Pastor:Gospel ministry ``f20!}\textit{ in accordance with the laws of God and of this province, I now pronounce you husband and wife}\index{wives ``f20!}\textit{. What therefore God has joined}\index{Marriage:joined together ``f20!}\textit{ together, let no man separate}\index{Marriage:no man separate ``f20!}\textit{.''}  

\begin{innerlst}[resume*]
      \item The ceremony may then conclude with a prayer for God's\index{Marriage:prayer:for God's blessing ``f20!} blessing. 
\end{innerlst} 

\subsection{Funerals} \index{Funeral:funeral ``f20!} 

\begin{innerlst}[resume*]
      \item A Christian funeral\index{Funeral:service ``f20!} should honour Jesus Christ and comfort the bereaved. Funerals\index{Funeral:funeral ``f20!} can be held in whatever place and at whatever time is most suitable. The Lord's Day\index{Funeral:avoid Lord's Day ``f20!} should ordinarily be avoided. 
      \item As there is a great difference between the end of a believer and\index{Funeral:believer's vs unbeliever's ``f20!} an unbeliever, the funeral\index{Funeral:service ``f20!} service may need to be modified. The suggested service that follows is designed with the believer in\index{Funeral:believer's ``f20!} mind. In every funeral service, the pastor is to point everyone present to Jesus Christ as the sole hope in life and in death\index{Funeral:present Jesus as sole hope in death ``f20!}. 
      \item There is to be no compromise with secret societies or\index{Funeral:no compromise with secret societies ``f20!} false religions. If called to officiate\index{Funeral:asked to officiate in non-Christian context ``f20!} where such a society desires to perform any of their rituals, the pastor should make his service distinctly separate, or else refuse\index{Funeral:refuse to participate in rituals of false religions ``f20!} to participate. Likewise, neither the pastor nor any Christian is to offer worship\index{Idolatry:false worship ``f20!} or veneration to any idol\index{Idolatry:veneration of idol ``f20!} or ancestor\index{Idolatry:ancestor worship ``f20!}. 
      \item Christians should mourn with those who mourn, yet not mourn as those without hope. Therefore, it is right and proper for Christians to gather with their families\index{Funeral:gather family ``f20!} and loved ones before and after funerals\index{Funeral:funeral ``f20!}, so long as their allegiance to the Lord Jesus\index{Funeral:practices not compromise allegiance to Lord Jesus ``f20!} is not compromised by any unbiblical practices, such as ancestor\index{Idolatry:ancestor worship ``f20!} worship or prayers\index{Idolatry:prayer for dead ``f20!} for, or to, the dead. 
      \item It is appropriate to consult the family about\index{Funeral:consult family ``f20!} suitable and comforting passages of Scripture. It is also appropriate to invite other\index{Funeral:invite other pastors to participate ``f20!} pastors of like precious faith to share in the service\index{Funeral:service ``f20!}. 
      \item The following is a suggested order for a funeral\index{Funeral:service ``f20!} service which may be varied: 
      \begin{itemize}[noitemsep]
            \item Greeting 
            \item Prayer of Adoration\index{Prayer:adoration ``f20!} and Invocation\index{Prayer:invocation ``f20!} 
            \item Singing\index{Psalms:singing:book of Psalms ``f20!} of a Psalm 
            \item Scripture reading  
            \item Prayer 
            \item Singing\index{Psalms:singing:book of Psalms ``f20!} of a Psalm 
            \item Scripture Reading\index{Worship:Scripture reading ``f20!} 
            \item Sermon 
            \item Singing\index{Psalms:singing:book of Psalms ``f20!} of a Psalm 
            \item Benediction\index{Worship:benediction:benediction ``f20!} 
      \end{itemize} 
      \item If a graveside gathering takes\index{Funeral:graveside gathering ``f20!} place, suitable Scripture may be read and expounded\index{Preaching:exposition of:Scripture or doctrine ``f20!}, and prayer offered for\index{Funeral:prayer offered for grieving ``f20!} those who are grieving. 
\end{innerlst} 

\item
\restartlist{innerlst}
\section{Chapter 4 -- Special Ordinances}  

\begin{innerlst}[resume*]
      \item The session shall seek to encourage these special ordinances\index{Worship:special ordinances ``f20!} towards the promotion of holiness of\index{Elders:promote holiness of life in congregation ``f20!} life and character of the congregation\footnote{\ 1 Timothy 4:7\index{54 1 Timothy:ch.   4, v.   7, pg. ``f30!}; Psalm 119:11\index{19 Psalms:ch. 119, v.  11, pg. ``f30!}.} as a means of bringing hope, comfort, joy, and revival to the soul\index{Elders:duties:care for souls ``f20!}. They may include prayer,\footnote{\ Matthew 6:5-9\index{40 Matthew:ch.   6, v.   5-9, pg. ``f30!}; Luke 5:16\index{42 Luke:ch.   5, v.  16, pg. ``f30!}; Luke 11:9\index{42 Luke:ch.  11, v.   9, pg. ``f30!}; Luke 18:1\index{42 Luke:ch.  18, v.   1, pg. ``f30!}; Colossians 4:2\index{51 Colossians:ch.   4, v.   2, pg. ``f30!}; 1 Thessalonians 5:17\index{52 1 Thessalonians:ch.   5, v.  17, pg. ``f30!}; Hebrews 4:16\index{58 Hebrews:ch.   4, v.  16, pg. ``f30!}.} reading\footnote{\ Matthew 4:4\index{40 Matthew:ch.   4, v.   4, pg. ``f30!}; Matthew 19:4\index{40 Matthew:ch.  19, v.   4, pg. ``f30!}.} and meditating\index{meditation on God's Word ``f20!}\footnote{\ Ezra 7:10\index{15 Ezra:ch.   7, v.  10, pg. ``f30!}; Proverbs 22:17-19\index{20 Proverbs:ch.  22, v.  17-19, pg. ``f30!}; Psalm 1:1-3\index{19 Psalms:ch.   1, v.   1-3, pg. ``f30!}; Psalm 119:97-99\index{19 Psalms:ch. 119, v.  97-99, pg. ``f30!}; Acts 17:11\index{44 Acts:ch.  17, v.  11, pg. ``f30!}; 2 Timothy 3:16\index{55 2 Timothy:ch.   3, v.  16, pg. ``f30!}; 2 Timothy 4:13\index{55 2 Timothy:ch.   4, v.  13, pg. ``f30!}; John 17:17\index{43 John:ch.  17, v.  17, pg. ``f30!}.} on God's Word\index{Bible:Word of God:to be meditated upon ``f20!}, and the singing\index{Psalms:singing:book of Psalms ``f20!} of Psalms.\footnote{\ Colossians 3:16\index{51 Colossians:ch.   3, v.  16, pg. ``f30!}; Ephesians 3:18-21\index{49 Ephesians:ch.   3, v.  18-21, pg. ``f30!}; Psalm 92:1-2\index{19 Psalms:ch.  92, v.   1-2, pg. ``f30!}; Psalm 98:1\index{19 Psalms:ch.  98, v.   1, pg. ``f30!}; Psalm 100:1\index{19 Psalms:ch. 100, v.   1, pg. ``f30!}; Psalm 106:1\index{19 Psalms:ch. 106, v.   1, pg. ``f30!}.}  
\end{innerlst} 

\subsection{Personal Worship} \index{Worship:private:personal ``f20!} 

\begin{innerlst}[resume*]
      \item Daily personal worship\index{Worship:private:personal ``f20!} is necessary because of the tendency of the human heart to yield to temptation\index{temptation ``f20!} and depart from the Lord.\footnote{\ James 1:13-15\index{59 James:ch.   1, v.  13-15, pg. ``f30!}.} It is also a means of disciplining ourselves for the purpose of godliness\index{Worship:discipline in godliness ``f20!}.\footnote{\ Romans 12:2\index{45 Romans:ch.  12, v.   2, pg. ``f30!}; Philippians 3:13, 14\index{50 Philippians:ch.   3, v.  13, 14, pg. ``f30!}; Timothy 4:7.} 
\end{innerlst} 

\subsection{Family Worship} \index{Worship:family:family ``f20!} 

\begin{innerlst}[resume*]
      \item Daily family\index{Worship:family:daily ``f20!} worship is essential to the development of family\index{Instruction:family religion ``f20!} religion and is a distinguishing mark of the Christian home.\footnote{\ Joshua 24:14, 15\index{06 Joshua:ch.  24, v.  14, 15, pg. ``f30!}; Acts 2:39\index{44 Acts:ch.   2, v.  39, pg. ``f30!}; 2 Timothy 1:5\index{55 2 Timothy:ch.   1, v.   5, pg. ``f30!}; 2 Timothy 3:15\index{55 2 Timothy:ch.   3, v.  15, pg. ``f30!}. } While devotional material\index{Instruction:devotional material:useful for understanding God's Word ``f20!} may be used to help with the understanding of God's Word\index{Instruction:devotional material:not replace God's Word ``f20!}, it should not replace the reading and\index{Worship:public:scripture reading ``f20!} meditation\index{meditation on God's Word ``f20!} in the Scriptures.\footnote{\ Psalm 1:2\index{19 Psalms:ch.   1, v.   2, pg. ``f30!}; Psalm 119:99\index{19 Psalms:ch. 119, v.  99, pg. ``f30!}; Philippians 4:8\index{50 Philippians:ch.   4, v.   8, pg. ``f30!}; James 1:25\index{59 James:ch.   1, v.  25, pg. ``f30!}.} It is the responsibility of the spiritual head of the home\index{Worship:family:led by spiritual head of home ``f20!} to ensure such worship\index{Worship:family:ensure such done ``f20!} takes place.\footnote{\ Genesis 8:20, 21\index{01 Genesis:ch.   8, v.  20, 21, pg. ``f30!}; Job 1:5\index{18 Job:ch.   1, v.   5, pg. ``f30!}; Joshua 24:14, 15\index{06 Joshua:ch.  24, v.  14, 15, pg. ``f30!}.} Thanksgiving\index{Prayer:thanksgiving ``f20!} for God's provision and prayer for God's\index{Worship:family:prayer for God's blessing ``f20!} blessing should also be offered before meals\index{Prayer:before meals ``f20!} are eaten.\footnote{\ Genesis 9:3\index{01 Genesis:ch.   9, v.   3, pg. ``f30!}; Matthew 14:19\index{40 Matthew:ch.  14, v.  19, pg. ``f30!}; Mark 8:7\index{41 Mark:ch.   8, v.   7, pg. ``f30!}; Luke 12:24\index{42 Luke:ch.  12, v.  24, pg. ``f30!}; John 6:23\index{43 John:ch.   6, v.  23, pg. ``f30!}; Acts 27:38\index{44 Acts:ch.  27, v.  38, pg. ``f30!}; 1 Thessalonians 5:16-18\index{52 1 Thessalonians:ch.   5, v.  16-18, pg. ``f30!}; 1 Timothy 4:4-5\index{54 1 Timothy:ch.   4, v.   4-5, pg. ``f30!}.} 
\end{innerlst} 

\subsection{Fellowship Groups} \index{Congregation:fellowship groups ``f20!}\index{Congregation:fellowship ``f20!} 

\begin{innerlst}[resume*]
      \item It is a blessing for the Church to gather during\index{Congregation:mid-week gatherings ``f20!} the week for mutual encouragement and edification.\footnote{\ Acts 20:20\index{44 Acts:ch.  20, v.  20, pg. ``f30!}; Romans 1:11, 12\index{45 Romans:ch.   1, v.  11, 12, pg. ``f30!}.} This may be a single gathering, or multiple gatherings in different locations and at different times. While devotional material\index{Instruction:devotional material:useful for understanding God's Word ``f20!} may be used to help with the understanding of God's Word\index{Instruction:devotional material:not replace God's Word ``f20!}, it should not replace the reading and\index{Worship:public:scripture reading ``f20!} meditation\index{meditation on God's Word ``f20!} in the Scriptures.\footnote{\ Psalm 1:2\index{19 Psalms:ch.   1, v.   2, pg. ``f30!}; Psalm 119:99\index{19 Psalms:ch. 119, v.  99, pg. ``f30!}; Acts 2:24\index{44 Acts:ch.   2, v.  24, pg. ``f30!}; Acts 17:11\index{44 Acts:ch.  17, v.  11, pg. ``f30!}; Acts 20:20\index{44 Acts:ch.  20, v.  20, pg. ``f30!}.}  
\end{innerlst} 

\subsection{Instruction Classes} \index{Instruction:classes:on Lord's Day ``f20!}\index{Congregation:instruction classes ``f20!} 

\begin{innerlst}[resume*]
      \item Instruction\index{Instruction:classes:on Lord's Day ``f20!} classes\index{Congregation:instruction classes ``f20!} for the congregation may be arranged by the session to take place on the Lord's Day\index{Sabbath:Lord's Day:worship ``f20!}. The purpose shall be to teach the doctrines\index{Instruction:classes:to teach doctrines of grace ``f20!} of grace and biblical truth set forth in the \textit{Westminster Confession}\index{Westminster:Confession of Faith ``f20!}\textit{ of Faith,} the \textit{Larger Catechism}\index{Westminster:Larger Catechism ``f20!}\textit{,} and the \textit{Shorter Catechism}\index{Westminster:Shorter Catechism ``f20!}.\footnote{\ Deuteronomy 6:4-9\index{05 Deuteronomy:ch.   6, v.   4-9, pg. ``f30!}; 2 Timothy 3:16\index{55 2 Timothy:ch.   3, v.  16, pg. ``f30!}; Acts 18:11\index{44 Acts:ch.  18, v.  11, pg. ``f30!}.} 
\end{innerlst} 

\end{outerlst}